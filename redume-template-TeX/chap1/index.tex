\thispagestyle{myheadings}

\section{はじめに}
\label{sec:intro}
研究室やコワーキングスペースのような場所では部屋利用者の在室情報が分かると様々な応用ができる.
部屋の利用者数や時間帯が把握できれば,環境整備や活用状況が少ない部屋の省エネ化の指標となる.
また目的とする人の居場所を把握できれば,コミュニケーションの円滑化や共同作業を支援できる.

しかし研究室のような場所では必ずしも在室情報が記録されているとは限らない.
またコアタイムが存在しないような研究室では常に活発なコミュニケーションがあるとは限らない
在室者を検出する方法としてスマートフォンやビーコンを用いた検出方法がある [1].スマートフォンとビーコンを利用し,在室者を検出する手法である.
しかし,部屋利用者が能動的に記録をしないといけないという問題点がある.また会社において気軽なコミュニケーション促進を目的とした研究がある [2].
しかし,システムの導入が会社におけるものなので研究室での環境に適合しないと考える.

そこで我々の先行研究として BLE ビーコンを用いた在室管理プラットフォーム「滞在ウォッチ」が提案されている.
滞在ウォッチでは利用者の負担軽減のために,在室者情報を BLEビーコンで受動的に記録する方法を採用されている.
在室管理プラットフォームの概要を図 1 に示す.ビーコンを持った利用者が部屋に訪れると受信機が検知し,サーバに在室者情報を送信しデータベースに記録する.
データベースに保存された情報は独自に作成したAPIによって外部からの利用が可能である.
過去の研究として滞在ウォッチAPIを用いた退勤管理システムや在室状況可視化システム,部屋利用者の来訪促進システム,コミュニケーション促進システムなど様々な応用システムの構築がされてきた.[4][5]

この滞在ウォッチの複数コミュニティ間での連携を考えている.
ここでいう複数コミュニティ間とは物理的な距離が近く,同じようなことをやっているコミュニティ間と定義する.
複数コミュニティ間で滞在ウォッチを連携したい理由としてこのようなコミュニティ間でコミュニケーション促進できれば知見の共有や新規性のある想像ができる可能性高いからである.
例として大学の研究室同士などが上げられる。

しかし滞在ウォッチは単一コミュニティでの運用が前提なため複数コミュニティ間で連携するには複数の問題点が存在する.
受信機のデータ精度が高くない,コミュニティで独立した運用ができていない,色んな属性の人が継続的に利用できない,継続的なメンテナンスが困難な点が上げられる.
上記の問題を解決しなければ複数のコミュニティ間で連携することは難しい.
これらの問題を解決した滞在ウォッチの運用を安定運用と定義する.
本研究では滞在ウォッチを複数コミュニティ間で連携するために独立したコミュニティにおける滞在ウォッチの安定運用のためのシステム拡張について提案する.