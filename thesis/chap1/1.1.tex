





\section{背景}\label{1.1}
研究室やコワーキングスペースのような場所では部屋利用者の在室情報が分かると様々な応用ができる.
在室者情報の利用方法は部屋を管理する管理者と利用する利用者では異なる.
まず,管理者の視点で考える.
部屋の利用者数や時間帯が把握できれば,室内の温度調整を始めとする環境整備や活用状況が少ない部屋の省エネ化の指標となる.
具体的には,人が集まりやすい時間帯は適した冷暖房設定にしたり,人がいなくなった時は必要のない家電の電源を切るといった対策を効果的に行える.
次に利用者の視点で考える.
目的とする人の居場所を把握できれば,接触までのアプローチが容易になり,コミュニケーションの円滑化や共同作業を支援できる.

しかし研究室のような場所では必ずしも在室情報が記録されているとは限らない.
部屋利用者のプライバシを考慮しなければならなかったり\cite{privacy},在室情報を記録するための導入コストが高いなど\cite{smartphone}の要因が考えられる.
また部屋利用者への負担を考えると導入に踏み切れない場合もある.




