





\section{背景}\label{1.1}
近年社会的に重要な課題の一つとして,在室者管理が挙げられる.
在室者管理は環境保護やエネルギー効率化セキュリティ強化などの目的から,家庭やオフィスなどの建物内での生活やビジネスに関する様々なアプリケーションに応用される.
例えばエネルギー管理は居住者がいる場合といない場合でエネルギー消費量が大きく異なるため,在室者管理によって建物内でのエネルギー消費を最適化できる.
またセキュリティや照明などのシステムの自動化にも応用され,これにより安全管理やエネルギー効率化を図れる.
在室者管理は災害時や緊急事態においても重要な役割を持つ.災害時には避難した居住者の確認が重要であり,在室者管理を使用すれば確認作業をスムーズに行える.

また在室者管理はコミュニティにおいても様々なメリットがある.その一つにコミュニティ内でのコミュニケーション促進が挙げられる.
在室者管理を行い在室者情報を可視化すれば,コミュニティ内で誰がいるのかを確認できる.これによりコミュニティ内での交流や活動がスムーズになり,同じコミュニティに属する人たちが同じ時間にいる場合には共同での過ごし方を提案できる.
またコミュニティ内でのイベントやミーティングの開催タイミングを調整できる.これにより参加者が多い時間帯にイベントやミーティングを開催できるため参加者が集まりやすくなり,コミュニケーションの促進を行える.

さらに近年在室者管理は新型コロナウイルスの影響により,感染拡大を防ぐ上で有効な手段と考えられる.
在室者管理システムを使用すれば,感染者が訪れた場所や接触した人々を追跡し消毒や清掃を行える.また濃厚接触者特定し隔離や検査を行えば早期発見や早期対応が可能になる.
そのためコミュニティなどで在室システムを導入すれば感染リスクを低減するために活用できると考えられる.

在室者管理の研究は,学術界や産業界においても注目を集めており,20世紀後半から様々な方法が提案されてきた.
在室者管理を行う方法としてICカードを用いて在室者情報を能動的に記録する方法がある.
例として学生が教室に入るときにICカードをタッチし,出るときにも同様にタッチを行うと学生が教室に在室した時間を記録できるシステムが挙げられる.
ICカードの使用により,手動での在室情報の管理から自動化が可能になる.
この方法では在室者情報を確実に記録できるがICカードを用いて能動的に記録する必要がある.
利用者への負担がICカードの記録忘れによる在室情報の不正確さを招く可能性がある.
この問題の解決策として,様々なセンサを使用した方法が提案されている.例えば,照度センサ,温度センサ,音声センサ,カメラなどが挙げられる.これらセンサを使用して得られたデータを処理し在室者がいるかどうかを判定する.
ただしこれらのセンサを使用した在室者管理は,環境条件や居住者の生活スタイルなどにより精度が異なるため,正確な在室者管理ができない場合がある.このように在室者管理は重要な課題であり様々なアプリケーションに応用されるが,環境によって正確な検出が難しいという問題もある.
また深層学習を使用した在室者管理システムがある.
深層学習は大量の画像や音声などのデータを特徴量として抽出し,学習を行うため高精度の人物の在室検出が可能である.
しかし深層学習を使用した在室者管理には,大量のデータが必要でありデータ収集や学習には多くのリソースが必要なのが問題点である.またプライバシ保護の観点から,カメラの使用は避けられる場合が多い.

我々の先行研究としてBLE(Bluetooth Low Energy)ビーコン(以下,ビーコン)を用いた在室者管理プラットフォーム「滞在ウォッチ」を提案している.
ビーコンの価格は低コストであり普及しやすいと考えられる.これにより建物全体に導入が容易であり在室者検出のカバー範囲を広げられる.またビーコンは低消費電力であり長期間にわたって使用できる.これにより,在室者検出を24時間行えるため建物内での生活やビジネスに対して常時モニタリングを行える.さらにビーコンはビーコン単体だけでは個人を特定できず,例え紛失したとしても個人を特定する情報を持たないためプライバシ保護にも適している.
滞在ウォッチはビーコンを持ち歩き在室情報を記録するメンバ,現在の状況や履歴を閲覧したりAPIを通して利用する利用者,メンバ管理,メンバへのビーコン配布,利用者の登録を行う管理者,システムを開発する開発者が存在する.メンバの負担軽減のため,
在室者情報をビーコンで受動的に記録する方法を採用している.ビーコンを持ったメンバが部屋に訪れると受信機が検知し,サーバに在室者情報を送信し,データベースに記録する.データベースに保存された情報は,独自に作成したWeb APIによって外部からの利用が可能である.先行研究として,Web APIを使用した退勤管理システムや在室情報可視化システム,部屋利用者の来訪促進システム,コミュニケーションプラットフォームなどがある.
%滞在ウォッチの概要を図\ref{StayWatchOverview}に示す.図\ref{StayWatchOverview}に示したように


