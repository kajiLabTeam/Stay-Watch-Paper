





\section{背景}\label{1.1}

% 研究室やコワーキングスペースのような場所では部屋利用者の在室情報が分かると様々な応用ができる.
% 在室者情報の利用方法は部屋を管理する管理者と利用する利用者では異なる.
% まず,管理者の視点で考える.
% 部屋の利用者数や時間帯が把握できれば,室内の温度調整を始めとする環境整備や活用状況が少ない部屋の省エネ化の指標となる.
% 具体的には,人が集まりやすい時間帯は適した冷暖房設定にしたり,人がいなくなった時は必要のない家電の電源を切るといった対策を効果的に行える.
% 次に利用者の視点で考える.
% 目的とする人の居場所を把握できれば,接触までのアプローチが容易になり,コミュニケーションの円滑化や共同作業を支援できる.
% また近年のコロナ禍において,利用者がいつ・どの部屋に滞在していたかという記録は重要となりうる.
% 具体的には,利用者がコロナ感染を発症した場合に,その利用者がどの部屋にいたかを記録しておけば,その部屋に滞在していた利用者の接触確認が容易になる.


% しかし研究室のような場所では必ずしも在室情報が記録されているとは限らない.
% 部屋利用者のプライバシを考慮しなければならなかったり\cite{privacy},在室情報を記録するための導入コストが高いなど\cite{smartphone}の要因が考えられる.
% また部屋利用者への負担を考えると導入に踏み切れない場合もある.
% そこで我々の先行研究としてBLEビーコンを在室管理プラットフォーム「滞在ウォッチ」が提案されている.


% 本研究は,研究室の部屋利用者の在室者情報を記録するプラットフォームと,プラットフォームによって記録された在室者情報を用いたシステムの提案をする.
% 研究室やコワーキングスペースのような場所では部屋利用者の在室情報が分かると様々な応用ができるが,必ずしも在室情報が記録されているわけではない.
% そこで我々は部屋利用者の在室情報が記録できる在室管理プラットフォーム「滞在ウォッチ」を提案する.
% 滞在ウォッチは在室者一人一人に携帯してもらったBLEビーコンの電波が,部屋に設置された受信機によって取得された時に在室情報を記録するものである.
% 記録された在室者情報は誰でも利用可能なWeb APIとして提供している.
% 記録された在室者情報を用いれば様々な応用システムが考えられる.
% その一つとして研究室でのコミュニケーションの機会損失を軽減,防ぐシステム「きょうの滞在」を提案する.
% きょうの滞在では現在の部屋利用者の在室情報を部屋に設置されているディスプレイに提示する.
% きょうの滞在で提示する情報として部屋利用者の顔写真とともに,来そう,今いる,帰りそう,帰ったという情報を付与する.
% 来そう,帰りそうは滞在ウォッチで記録した在室者情報をもとに推定を行う.
% その推定結果が正確であるかの評価を行う.

近年,社会的に重要な課題の一つに,在室者管理の課題がある.
在室管理は,環境保護やエネルギー効率化,セキュリティ強化などの目的から,家庭やオフィスなどの建物内での生活やビジネスに関する様々なアプリケーションに応用される.
例えば,エネルギー管理は,居住者がいる場合といない場合でエネルギー消費量が大きく異なるため,在室者管理を使用して,建物内でのエネルギー消費を最適化できる.
また,セキュリティや照明などのシステムの自動化にも応用され,これにより,環境保護やエネルギー効率化を図れる.
在室者管理は,災害時や緊急事態においても重要な役割を持つ.災害時には,避難した居住者を確認が重要であり,在室者管理を使用すれば,確認作業をスムーズに行える.

また在室者管理は,コミュニティにおいても様々なメリットがある.その一つに,コミュニケーション促進が挙げられる.
在室者管理を行い在室者情報を可視化すれば,コミュニティ内で誰がいるのかを確認できる.これにより,コミュニティ内での交流や活動がスムーズになり,共同生活をする人たちが同じ時間にいる場合には,共同での食事や過ごし方を提案でき,コミュニケーションを促進できる.
また,コミュニティ内でのイベントやミーティングの開催タイミングを調整できる.これにより,参加者が多い時間帯に開催できるため,参加者が集まりやすくなり,コミュニケーションの促進を行える.


また近年,在室者管理は新型コロナウイルスの影響により,感染拡大を防ぐ上で有効な手段と考えられている.
在室管理システムを使用すれば,建物内の感染者を発見し,隔離できる.システムは,感染者が訪れた場所や接触した人々を追跡,
感染が広がった可能性が高い人々を特定し,隔離や検査を行えるため,早期発見・早期対応ができる.
感染者の位置情報が判明しているため,消毒や清掃を行える.
そのために,病院や医療機関などでは,在室者管理システムを導入し,患者やスタッフの感染リスクを低減するために活用されている.

在室者管理の研究は,学術界や産業界においても注目を集めており,20世紀後半から様々な方法が提案されてきた.
近年では,様々なセンサを使用した方法が提案されている.例えば,照度センサや温度センサ,音声センサ,カメラなどが使用される.これらセンサを使用して得られたデータを処理し,居住者がいるかどうかを判定する.
ただし,これらのセンサを使用した在室者管理は,環境条件や居住者の生活スタイルなどにより精度が異なるため,正確な在室者管理ができない場合がある.また,プライバシー保護の観点から,カメラを使用は避けられる場合が多い.このように,在室者管理は重要な課題であり,様々なアプリケーションに応用されるが,環境やプライバシーにより,正確な検出が難しいという問題もある.
他の在室者管理に関する方法として深層学習を使用した在室者管理がある.深層学習は,大量のデータを学習し,高い精度を得られるため,在室者管理においても有効であると考えられている.
また,在室者管理においては,プライバシー保護も重要な観点である.深層学習を使用すレバ,画像や音声などのデータを特徴量として抽出し,学習を行うため,プライバシーに敏感な情報を取り込まない.
しかしながら,深層学習を使用した在室者管理には,大量のデータが必要であり,データ収集や学習には多くのリソースが必要なのが問題点である.

我々の先行研究としてBLE(Bluetooth Low Energy)ビーコン(以下,ビーコン)を用いた在室管理プラットフォーム「滞在ウォッチ」を提案している
ビーコンは低コストであり,普及しやすいと考えられる.これにより,建物全体に導入が容易であり,在室者検出のカバー範囲を広げるられる.また,ビーコンは低消費電力であり,長期間にわたって使用できる.これにより,在室者検出を24時間行えるため,建物内での生活やビジネスに対して常時モニタリングを行える.さらに,ビーコンはプライバシー保護にも優れている.ビーコンは個人を特定できないため,個人情報の保護にも適している.
滞在ウォッチは,ビーコンを持ち歩き在室情報を記録するメンバ,現在状況や履歴を閲覧したりAPIを通して利用する利用者,メンバ管理,メンバへのビーコン配布,利用者の登録を行う管理者,システムを開発する開発者が存在する.メンバの負担軽減のため,
室者情報をビーコンで受動的に記録する方法を採用している.ビーコンを持った利用者が部屋に訪れると受信機が検知し,サーバに在室者情報を送信し,データベースに記録する.データベースに保存された情報は,独自に作成したWeb APIによって外部からの利用が可能である.先行研究として,Web APIを使用した退勤管理システムや在室情報可視化システム,部屋利用者の来訪促進システム,コミュニケーションプラットフォームなどがある.
%滞在ウォッチの概要を図\ref{StayWatchOverview}に示す.図\ref{StayWatchOverview}に示したように


