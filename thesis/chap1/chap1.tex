\chapter{はじめに}
\thispagestyle{myheadings}

卒業研究は1名(個人)または2〜4名のグループで行う.卒業研究を行う個人
またはグループは以降に示すようにレジュメの執筆,本論文の執筆,口頭発表
を行わなければならない.なお,グループで卒業研究を行う場合は,グループ
で1件,執筆,発表等を行うことになる.

\section{スケジュール}
\label{sec:schedule}

スケジュールはL-Cam等を使って連絡されるので注意する.
\begin{description}
  \itemindent=1zw
  \itemsep=0mm
        \parsep=0mm
  \item[*月**日(*)] タイトル提出
  \item[*月**日(*)] レジュメ提出
  \item[*月**日(*)] 卒業論文締切
  \item[*月**日(*)] 卒業論文・制作発表会
\end{description}

\section{レジュメの提出}
\label{sec:abstract}

先に示した締切にしたがって,卒業研究の要旨を記したものを提出する.これ
をレジュメという.レジュメは,指導教員の校閲を受け,提出許可を受けてか
ら,指導教員経由で電子的に提出する.

レジュメは,A4サイズにて提出し,印刷されて,\ulinej{審査にあたる各教員
  と4年生及び各研究室に配布される.口頭発表時には,教員や聴講している学生は
  このレジュメを見ながら質問をする.}
よく推敲して,研究の要旨を充分に伝えられるものに仕上げる.
指導教員から提出を認められるまで,何度でも書き直す.
レジュメの作成にあたっては \LaTeX やWordなどのソフトウェアを使用する.

\section{卒業論文提出}
\label{sec:thesis}

卒業研究の成果は,指導教員の指示に従って,論文の形式にまとめ,それを必
要部数製本する(通常2部).論文は,情報科学科事務室に1部提出する.
提出された論文は,専攻で審査に使われた後,各教員の元で永年にわたって保
存される.卒業論文の提出にあたっては,レジュメの場合と同様に,指導教員
の校閲を受け,また認印を(内表紙に)受ける.

卒業論文のおおよその書式は第2章で説明するが,各研究室のスタイルに従え
ば良い.\LaTeX を利用する場合は,このマニュアルの\LaTeX ソースをテンプレー
トと利用しても良い.

\section{卒業研究口頭発表}
\label{sec:presentation}

卒業研究の仕上げは,卒業研究口頭発表である.卒業研究を1人で行っている
人は個人で,2人以上のグループで行っている人はグループで発表する.
口頭発表は,公開制であり,1〜3年生の聴講も許されている.

口頭発表技術は,話し方,資料のまとめ方,印象的な資料の提示方法等も含む.
口頭発表技術を,ぜひこの機会を利用して習得して欲しい.発表はプレゼンテー
ションソフトを利用して,ノートパソコンと液晶プロジェクターの組み合わせ
で効果的な発表を行う.発表時間は次を基準とする.\\

\begin{tabular}[h]{ll}
  1人で発表を行う場合      & 10分(質疑応答3分を含む) \\
  グループで発表を行う場合 & 15分(質疑応答5分を含む) \\
\end{tabular}

\section{卒業研究の審査・単位認定について}
\label{sec:referee}
審査は論文,レジュメ,口頭発表および日常の研究態度を総合して行われる.

% Local Variables: 
% mode: japanese-LaTeX
% TeX-master: "root"
% End: 
