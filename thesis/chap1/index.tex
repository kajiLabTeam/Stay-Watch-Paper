\thispagestyle{myheadings}


\chapter{はじめに}\label{1}
本研究は,研究室の部屋利用者の在室者情報を記録するプラットフォームと,プラットフォームによって記録された在室者情報を用いたシステムの提案をする.
研究室やコワーキングスペースのような場所では部屋利用者の在室情報が分かると様々な応用ができるが,必ずしも在室情報が記録されているわけではない.
そこで我々は部屋利用者の在室情報が記録できる在室管理プラットフォーム「滞在ウォッチ」を提案する.
滞在ウォッチは在室者一人一人に携帯してもらったBLEビーコンの電波が,部屋に設置された受信機によって取得された時に在室情報を記録するものである.
記録された在室者情報は誰でも利用可能なWeb APIとして提供している.
記録された在室者情報を用いれば様々な応用システムが考えられる.
その一つとして研究室でのコミュニケーションの機会損失を軽減,防ぐシステム「きょうの滞在」を提案する.
きょうの滞在では現在の部屋利用者の在室情報を部屋に設置されているディスプレイに提示する.
きょうの滞在で提示する情報として部屋利用者の顔写真とともに,来そう,今いる,帰りそう,帰ったという情報を付与する.
来そう,帰りそうは滞在ウォッチで記録した在室者情報をもとに推定を行う.
その推定結果が正確であるかの評価を行う.







\section{背景}\label{1.1}

研究室やコワーキングスペースのような場所では部屋利用者の在室情報が分かると様々な応用ができる.
在室者情報の利用方法は部屋を管理する管理者と利用する利用者では異なる.
まず,管理者の視点で考える.
部屋の利用者数や時間帯が把握できれば,室内の温度調整を始めとする環境整備や活用状況が少ない部屋の省エネ化の指標となる.
具体的には,人が集まりやすい時間帯は適した冷暖房設定にしたり,人がいなくなった時は必要のない家電の電源を切るといった対策を効果的に行える.
次に利用者の視点で考える.
目的とする人の居場所を把握できれば,接触までのアプローチが容易になり,コミュニケーションの円滑化や共同作業を支援できる.

しかし研究室のような場所では必ずしも在室情報が記録されているとは限らない.
部屋利用者のプライバシを考慮しなければならなかったり\cite{privacy},在室情報を記録するための導入コストが高いなど\cite{smartphone}の要因が考えられる.
また部屋利用者への負担を考えると導入に踏み切れない場合もある.





\section{滞在ウォッチの安定運用のためのシステム拡張}\label{1.2}

本研究の目的は滞在ウォッチによる複数コミュニティ間連携の実現である.
% 滞在ウォッチは背景で述べたようにコミュニティなどにおける特定の場所にいる人々を把握するための装置やシステムである
複数コミュニティ間とは空間的な距離が近いコミュニティ間と定義する.
例として大学の研究室間やビルのオフィス間などが挙げられる.
滞在ウォッチにおける複数コミュニティ間連携によって得られるメリットとして,まずコミュニティ間でのコミュニケーションの増加が挙げられる.
普段コミュニケーションを取らない別コミュニティの人々とのコミュニケーションが促進される.
他のコミュニティからの新しい知見やアイデアの共有による知識量や創造性の向上,
スムーズなコミュニケーションやリソースの共有による生産性の向上が期待できる.
またコミュニティ間の連携が容易になるため,新しいプロジェクトやイベントが生まれる可能性がある.

% 滞在ウォッチにおける複数コミュニティ間連携によって,普段コミュニケーションを取らない別のコミュニティスペースにいる在室者を把握できる.
% コミュニケーションを促進できれば,知見の共有や新規性のある想像ができる可能性が高い.
しかし滞在ウォッチを複数コミュニティ間で連携するには大きく分けて2つの問題点が存在する.
まずコミュニティで独立した運用ができていない点が挙げられる.
「独立した運用」とは,各コミュニティで自らのシステムや装置を運用できる状態を指す.
複数コミュニティ間で連携するためには,各コミュニティで滞在ウォッチを運用する必要がある.
これを行うには各コミュニティが設備やシステムを持つ必要がある.
それぞれのコミュニティが独立した設備やシステムを持っていれば,各コミュニティが独自に在室情報を収集し
管理できるためそれらを連携させられる.
しかし既存の滞在ウォッチは単一コミュニティを前提として設計されているため各コミュニティが独立した運用を行えない.
次に在室情報を長期に渡り継続的に記録できない点が挙げられる.在室情報の長期に渡る継続的な記録は,コミュニティ間でのコミュニケーションを促進するための基礎となる情報であるため重要である.
しかし,既存の滞在ウォッチはBLEビーコンのみを使用して運用しているためバッテリがなくなると在室情報を記録できない.
バッテリの交換を促しているが交換がされず放置される場合がある.

これらの問題を解決した滞在ウォッチの運用を安定運用と定義し本研究では滞在ウォッチを複数コミュニティ間で連携するために,独立したコミュニティにおける滞在ウォッチの安定運用のためのシステム拡張について提案する.
安定運用に向けたアプローチとして既存システムの再構築,独立した運用に向けて利用者の管理とアクセス制御システムの整備,
長期に渡る継続的な在室情報の記録に向けてBLEビーコンのアプリケーション化とアプリケーション化を行ったスマホビーコンと実デバイスビーコンによるハイブリットシステムの構築を行う.






\section{論文構成}\label{1.3}
第\ref{2}章では,本研究と関連した研究との違いを比較する.
第\ref{3}章では,在室管理プラットフォーム「滞在ウォッチ」について述べる.
第\ref{4}章では,独立したコミュニティにおける滞在ウォッチの安定運用のためのシステム拡張について述べる.
第\ref{5}章では本研究に対するまとめと今後の課題について述べる.
\thispagestyle{myheadings}

