\section{滞在ウォッチの安定運用のためのシステム拡張}\label{1.2}
% そこで我々は部屋利用者の在室情報が記録できる在室管理プラットフォームの構築を目指す.
% 部屋利用者の在室者情報を部屋利用者が意識せずに記録され,記録された在室者情報を誰でも利用できるようなシステムを目指す.

% 我々はBLE(Bluetooth Low Energy)ビーコン(以下,ビーコン)\cite{beacon}を用いた在室管理プラットフォーム「滞在ウォッチ」を提案する.
% 本研究では手軽さと利用者の負担軽減のために,在室者情報をビーコンで受動的に記録する方法を採用した.%滞在ウォッチの概要を図\ref{StayWatchOverview}に示す.図\ref{StayWatchOverview}に示したように
% ビーコンを持った利用者が部屋に訪れると受信機が検知し,サーバに在室者情報を送信しデータベースに記録する.
% データベースに保存された情報は独自に作成したWeb APIによって外部からの利用が可能である.
% Web APIによって退勤管理システムや在室情報可視化システム,部屋利用者の来訪促進システム,コミュニケーション促進システムなど様々な応用システムの構築ができる.

% また研究室でのコミュニケーションの機会損失を軽減およびコミュニケーションが誘発されるようなシステムの構築を目指す.
% 研究室でのコミュニケーションは必要な要素の一つである.
% 研究室に所属する人同士の議論であったり,研究のアイデア出しなどが挙げられる.
% そのためコミュニケーションの機会損失を軽減,防ぐシステムが必要である.
% コアタイムがないような研究室ではコミュニケーションの機会が常にあるとは限らない.
% コアタイムがなければ人が集まりやすい状況が少なくなりコミュニケーションが発生しづらくなる.
% そのような状況が続くとコミュニケーションを取ろうという機会も少なくなる.
% また研究室のような共有空間では必ずしも共通の話題を持った人たちが集まるわけではない.
% 共通の話題がない場合コミュニケーションが発生しづらくなってしまう.
% 上記のような問題を解決するようなシステムを構築する.

% 研究室でのコミュニケーションの機会損失を軽減および気軽なコミュニケーションを誘発するために,ディスプレイに在室者情報を提示するシステム「きょうの滞在」を提案する.
% 部屋利用者の推定を来そう,今いる,帰りそう,帰ったの4種類に分類し,部屋利用者の顔写真を提示する.
% 滞在ウォッチが取得する在室者情報には部屋利用者が現在在室しているかどうか,入室時刻,退室時刻などがある.
% 現在在室しているかどうかの情報を用いれば今いる,帰ったという分類ができる.
% 入室時刻の情報を用いれば,特定の人物が入室しそうな時刻を計算できる.
% 退室時刻の情報を用いれば,特定の人物が退室しそうな時刻が計算できる.
% 計算した結果から来そう,帰りそうの分類ができる.
% 研究室に設置してあるディスプレイに部屋利用者の現在の在室情報を提示する.
% 来そう,帰りそうの表示によってコミュニケーションの機会損失を軽減できるのではないかと考える.
% また提示されている在室者情報を見たときにコミュニケーションが発生するのではないかと考えられる.




本研究の目的は,滞在ウォッチによる複数コミュニティ間連携の実現である.
% 滞在ウォッチは背景で述べたようにコミュニティなどにおける特定の場所にいる人々を把握するための装置やシステムである
複数コミュニティ間とは空間的な距離が近いコミュニティスペース間と定義する.
例として,大学の研究室間やビルのオフィス間などが挙げられる.
滞在ウォッチにおける複数コミュニティ間連携によって得られる利点として,まずコミュニティ間でのコミュニケーションの増加が挙げられる.
普段コミュニケーションを取らない別コミュニティの人々とのコミュニケーションが促進される.
次に共有や,生産性の向上が挙げられる.他のコミュニティから新しい知見やアイデアを共有することができるため,想像量や創造性が向上する,
スムーズなスムーズなコミュニケーションやリソースを共有できるため,生産性が向上する可能性がある.
またコミュニティ間のコラボレーションが容易になるため,新しいプロジェクトやイニシアティブが生まれる可能性がある.

% 滞在ウォッチにおける複数コミュニティ間連携によって,普段コミュニケーションを取らない別のコミュニティスペースにいる在室者を把握できる.
% コミュニケーションを促進できれば,知見の共有や新規性のある想像ができる可能性が高い.
しかし滞在ウォッチを複数コミュニティ間で連携するには大きく分けて2つの問題点が存在する.
まずコミュニティで独立した運用ができていない点が挙げられる.
「独立した運用」とは,各コミュニティで自らのシステムや装置を運用できることを指している.
複数コミュニティ間で連携するためには,各コミュニティで滞在ウォッチを運用するために必要な設備やシステムを持つ必要がある.
それぞれのコミュニティが独立したシステムを持っているいれば,各コミュニティが独自に滞在情報を収集し,
管理するできるため,それらを連携させられる.
しかし既存の滞在ウォッチは単一コミュニティを前提として設計されているため各コミュニティが独立した運用を行えない.
次に在室情報を長期に渡り継続的に記録できない点が挙げられる.在室情報の長期的な記録は,コミュニティ間でのコミュニケーションを促進するために必要である.
既存の滞在ウォッチで使用されるBLEビーコンのみを使用して運用しているため電池がなくなると在室情報を記録できない.
これらの問題を解決した滞在ウォッチの運用を安定運用と定義する.
本研究では滞在ウォッチを複数コミュニティ間で連携するために,独立したコミュニティにおける滞在ウォッチの安定運用のためのシステム拡張について提案する