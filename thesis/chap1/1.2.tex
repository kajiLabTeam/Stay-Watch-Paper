\section{滞在ウォッチの安定運用のためのシステム拡張}\label{1.2}

本研究の目的は滞在ウォッチによる複数コミュニティ間連携の実現である.
% 滞在ウォッチは背景で述べたようにコミュニティなどにおける特定の場所にいる人々を把握するための装置やシステムである
複数コミュニティ間とは空間的な距離が近いコミュニティ間と定義する.
例として大学の研究室間やビルのオフィス間などが挙げられる.
滞在ウォッチにおける複数コミュニティ間連携によって得られる利点として,まずコミュニティ間でのコミュニケーションの増加が挙げられる.
普段コミュニケーションを取らない別コミュニティの人々とのコミュニケーションが促進される.
他のコミュニティから新しい知見やアイデアを共有による想像量や創造性の向上,
スムーズなコミュニケーションやリソースを共有による生産性の向上が期待できる.
またコミュニティ間の連携が容易になるため,新しいプロジェクトやイベントが生まれる可能性がある.

% 滞在ウォッチにおける複数コミュニティ間連携によって,普段コミュニケーションを取らない別のコミュニティスペースにいる在室者を把握できる.
% コミュニケーションを促進できれば,知見の共有や新規性のある想像ができる可能性が高い.
しかし滞在ウォッチを複数コミュニティ間で連携するには大きく分けて2つの問題点が存在する.
まずコミュニティで独立した運用ができていない点が挙げられる.
「独立した運用」とは,各コミュニティで自らのシステムや装置を運用できることを指す.
複数コミュニティ間で連携するためには,各コミュニティで滞在ウォッチを運用する必要がある.
これを行うには必要な設備やシステムを持つ必要がある.
それぞれのコミュニティが独立した設備やシステムを持っていれば,各コミュニティが独自に在室情報を収集し
管理できるためそれらを連携させられる.
しかし既存の滞在ウォッチは単一コミュニティを前提として設計されているため各コミュニティが独立した運用を行えない.
次に在室情報を長期に渡り継続的に記録できない点が挙げられる.在室情報の長期に渡る継続的な記録は,コミュニティ間でのコミュニケーションを促進するための基礎となる情報であるため必要である.
しかし,既存の滞在ウォッチはBLEビーコンのみを使用して運用しているため電池がなくなると在室情報を記録できず,電池の交換を促しているが交換がされず放置される場合がある.

これらの問題を解決した滞在ウォッチの運用を安定運用と定義し本研究では滞在ウォッチを複数コミュニティ間で連携するために,独立したコミュニティにおける滞在ウォッチの安定運用のためのシステム拡張について提案する.
安定運用に向けたアプローチとして既存システムの再構築,独立した運用に向けて利用者の管理とアクセス制御システムの整備,
長期に渡る継続的な在室情報の記録に向けてBLEビーコンのアプリケーション化とアプリケーション化を行ったスマホビーコンと実デバイスビーコンによるハイブリットシステムの構築を行う.





