




サーバ側には独立したバックエンドシステムとの連携を容易でありWebアプリケーションがバックエンドシステムとの親和性が高く、
より高い可用性とスケーラビリティを実現するができるREST APIを採用した.REST APIは、
複数のクライアントからアクセスができるため,様々なデバイスやプラットフォームからアクセスできる。これにより、より広いユーザー層からアクセスができる
既存の滞在ウォッチのサーバ側のシステムはpythonを用いて構築されており,動的型付け言語なため保守性が低いものであった.
そこで静的型付けであり,高速な処理能力と小さなメモリフットプリントを持つため、Web APIの開発に適しているGolangを採用した.
またGolangは並列処理を容易に実現できるため、高負荷な環境でのWeb APIの開発にも適している。さらに、標準パッケージによるHTTPサーバのサポートを持つため、
Web APIの開発に必要な機能を簡単に実装できる。
これらの特徴から、Golangを使用したWeb APIの開発は、高速でスケーラブルなAPIを提供することができ、開発効率も高い