



\subsection{サーバの再設計と再構築}\label{4.1.1}
サーバ側には独立したバックエンドシステムとの連携が容易でありWebアプリケーションがより高い可用性とスケーラビリティを実現できるREST APIを採用した.
REST APIは、複数のクライアントからアクセスができるため,様々なデバイスやプラットフォームからアクセスできる.
これにより、より広いユーザー層からアクセスが期待できる.


滞在ウォッチのサーバ側システムはPythonで構築されていたが、高負荷環境でパフォーマンスに影響を与える可能性があったのと動的型付け言語の開発,保守のしづらさを解消するため、Go言語を使用してシステムの再構築を行った.
既存の滞在ウォッチのサーバ側のシステムはPythonを用いて構築されていた.Web APIは大量の処理をする必要があり,
Pythonの実行速度の遅さが,高負荷環境でパフォーマンスに影響を与える可能性がある.
またデフォルトで非同期プログラミンをサポートしていないため,
非同期処理に対応するために追加のライブラリを使用する必要がある.さらに動的型付け言語なため,型の不明確さによってバグが見つけづらく保守性が低い.
そこで静的型付けであり,高速な処理能力と小さなメモリフットプリントを持つため、Web APIの開発に適しているGo言語を採用した.
Go言語は高いパフォーマンスを持つ.C言語のような低レベルな言語と
同等のパフォーマンスを持ちながら,高水準の言語のような簡単な構文を持っているのに加え,並列処理を容易に実現できるため、
高負荷な環境でのWeb APIの開発にも適している。
そのため,WebAPIの要求に適合するために必要なパフォーマンスを提供できる。
また構造体とインターフェース型を備えており,WebAPIの開発に必要な柔軟性を持っている.これにより,開発者は,WebAPIを実装するために適した方法を選択できる.さらに、標準パッケージによるHTTPサーバのサポートを持つため、Web APIの開発に必要な機能を簡単に実装できる。

サーバで使用するデータベースには無料で利用でき,高いスケーラビリティを持つため,WebAPIの要求に対応可能なMySQLを採用した.MySQLは多くのプラットフォームでサポートされており,多くの言語に対応しており,開発者にとって選択肢が広がる.またMySQLは優れたセキュリティ機能を持つため、WebAPIのセキュリティの確保ができる.

Go言語でMySQLの操作を行う際Go言語のORMライブラリであるGormを採用した.
ORMはデータベースを操作を行うための手法の一つである.これにより,データベース操作を行う際に,SQL文を直接記述する必要がなくなり,コードがすっきりし,可読性が高くなる.
またSQL分を直接記述しないためSQLインジェクション対策にもなる.SQLインジェクションとは不正なSLQ文の挿入を行い,データベースに対して攻撃を行う手法である.ORMは,プログラマが入力したデータを自動的にエスコープし,サイバー攻撃につながるような文字を無効化するためSQLインジェクション攻撃を防止できる.
GormはGo言語向けORMライブラリの一つである.Gormはデータベースの挿入,更新,削除,検索などの基本的な機能操作をサポートしている.また自動的なマイグレーション機能を持っている.これにより開発者はデータベーススキーマを手動で管理せずに,スキーマの変更を自動的に反映できる.
GoのORMライブラリとして他にXormなどの選択肢もあったが複雑なクエリを簡単に作成するための豊富なクエリビルダーAPIを提供しているためGormを選択した.


サーバサイドプログラムの開発及びデプロイメントする際にはプログラムを実行するために必要な全ての環境をコンテナ内にパッケージ行い,
開発環境と本番環境での環境差を最小限に抑えられるDockerを採用した.
Dockerを使用すれば、アプリケーションを実行するために必要な全ての依存関係を詰め込んだイメージを作成できる.これにより、開発者は、アプリケーションを実行するために必要な環境を1つのイメージにまとめられ、それを任意の環境に簡単に展開でき開発環境の構築が容易になる.
さらにDockerは他のアプリケーションの影響を受けずに実行可能である.
これにより、アプリケーションが依存しているライブラリやフレームワークのバージョンなどの環境の違いによって、アプリケーションが動作しないといった問題を回避できる.
またコンテナを水平にスケールアウトできるため,アクセスが集中した場合にも,スムーズな処理を行える.
デプロイ先は,研究室内のサーバを利用した.研究室内のサーバの利用はクラウドサービスに比べて,長期的にコストを削減できる.





