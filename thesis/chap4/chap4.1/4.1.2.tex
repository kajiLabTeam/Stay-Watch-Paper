

\subsection{クライアントサイドの実装}\label{4.1.2}
クライアントサイドでは,WebAPIを利用して,Webアプリケーションを実装した.
Webアプリケーションのフレームワークとして,Next.jsを採用した.
Next.jsはReact.jsをベースにしており,React.jsの開発を効率化に加え,サーバーサイドによるレンダリングをサポートによるSEO対策やパフォーマンスの向上を実現できる.
React.jsは,JavaScriptのライブラリであり,コンポーネントベースのアーキテクチャを採用している.
コンポーネントを独立した単位として開発すると,開発者はそれぞれのコンポーネントに集中できるため,開発効率が向上する.
コンポーネントは再利用可能な単位であり,複数のページで同じコンポーネントを利用することができ,これにより,開発コストを削減し,保守性を向上させる.
またReactは,仮想DOMとは,実際のDOM(Document Object Model)を操作する代わりに、JavaScriptのオブジェクトとして扱う仕組みを採用している.
仮想DOMは,実際のDOMと比較して,更新が必要な部分だけを計算を行うため,パフォーマンスの向上が図れる.











