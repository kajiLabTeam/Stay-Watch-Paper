

\subsection*{4.1.2 クライアントサイドの実装}
クライアントサイドでは,WebAPIを利用して,Webアプリケーションを実装した.
Webアプリケーションのフレームワークとして,Next.jsを採用した.Next.jsはReact.jsをベースにしたフレームワークである.
Next.jsはReact.jsの開発を効率化するためのフレームワークである.Reactは,JavaScriptライブラリであり,コンポーネントベースのフレームワークである.
Reactは,コンポーネントベースのフレームワークであるため,開発者は,コンポーネントを組み合わせれば,複雑なWebアプリケーションを構築できる.
また,Reactは,仮想DOMと呼ばれる仕組みを利用しているため,DOMの操作を行う際に,DOMの再描画を最小限に抑えられる.
これにより,Webアプリケーションのパフォーマンスを向上させられる.





