\subsection{スマートフォンビーコンと実デバイスによるビーコンのハイブリッド活用}
スマートフォンビーコンのみを利用する場合,様々な状況下でメンバの継続した利用が困難であるため,実デバイスによる
ビーコンと併用できるシステムとした.
長期にわたり継続的に利用するメンバにとっては,実デバイスによるビーコンは先述の通りバッテリ交換の手間がある.
メンバはその手間からバッテリ交換をしないで放置する場合がある.その場合は在室情報が記録されず可用性が低下する.
スマートフォンビーコンはバッテリ交換を手間に考えて放置するメンバにとっては,バッテリ交換の手間がないため有用である.
しかしスマートフォンを携帯していないメンバ,スマートフォンビーコンの利用に伴うバッテリ消費が気になるメンバなども想定される.

実デバイスによるビーコンとスマートフォンビーコン双方にメリットデメリットが存在する.
それらはメンバの環境によって重要性が異なる.
例えばスマートフォンビーコンを利用しないメンバとして
スマートフォンに我々のアプリケーションを入れたくないメンバやスマートフォンがアプリに対応していないメンバ,
実デバイスによるビーコンを新しく持ち歩きを不安に感じるメンバなどである.
メンバの思想や所有している端末の都合など様々な要因が存在しており,それらをすべて満たした仕様の策定は困難である.
よってスマートフォンビーコンと実デバイスによるビーコンのハイブリッド化によってメンバの環境に応じた選択を可能にした.
様々な環境の中でも選択肢を提供し,ユーザの利便性を向上することでメンバのより積極的な参加を促すことができデータの可用性向上につながると考えた.

ハイブリッド化にあたってスマホビーコンで利用するUUIDを実デバイスによるビーコンで利用するUUIDと同じ値に設定しメンバの在室情報を記録している.
スマートフォンビーコンと実デバイスによるビーコン双方をメンバが受け入れて同時に利用した場合,
同じUUIDを利用しているため片方が停止した場合でも,もう片方で補完が可能である.
この方法は,メンバの利便性を向上できるのみならず,継続的にデータを記録する観点から見ても有用である.
実デバイスによるビーコン,スマートフォンビーコンの単独対応とハイブリッド化を表\ref{fig:hybrid}で比較する.
 滞在ウォッチが実デバイスによるビーコンのみに対応していた場合は
長期にわたり継続的に利用するメンバにとってはビーコンの電池交換が手間である.
短期的に利用するメンバにとってはビーコンを携帯するだけで済み手軽である.
しかしビーコンを不注意などで携帯していなかった場合はシステムに記録されない.
スマートフォンビーコンのみに対応していた場合は
長期に渡り継続的に利用するメンバにとってはアプリは手間が少ない.
短期的に利用しするメンバにとってはアプリケーションの登録は手間である.
しかし,スマートフォンが利用できない状態ではシステムに記録されない.
ハイブリッド化をした場合は,それぞれのユースケースに対応できる.

\begin{table}[tbh]
  \caption{各ビーコンのみ対応時とハイブリッド対応時の比較}
  \begin{tabular}{|c|c|c|c|c|}
    \hline
                              & 長期  メンバ              & 短期メンバ              & スマホ不携帯 & ビーコン不携帯 \\ \hline
    \begin{tabular}[c]{@{}c@{}}実デバイス\\ ビーコンのみ\\\end{tabular} & \begin{tabular}[c]{@{}c@{}}\\△\\ バッテリ交換の手間\end{tabular} & ○                         & ○            & ×              \\ \hline
    \begin{tabular}[c]{@{}c@{}}スマホ\\ ビーコンのみ\\\end{tabular} & ○                         & \begin{tabular}[c]{@{}c@{}}\\△\\ インストールの手間\end{tabular} & ×            & ○              \\ \hline
    \begin{tabular}[c]{@{}c@{}}ハイブリッド\\ 対応\\\end{tabular} & \begin{tabular}[c]{@{}c@{}}\\○\\\\\end{tabular} & ○                         & ○            & ○              \\ \hline
  \end{tabular}
  \label{fig:hybrid}
\end{table}