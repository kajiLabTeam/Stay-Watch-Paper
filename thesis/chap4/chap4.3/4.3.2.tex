\subsection{スマホビーコンと実デバイスによるビーコンのハイブリッド活用}
スマホビーコンのみを利用する場合,様々な状況下でメンバの継続した利用が困難であるため,実デバイスによる
ビーコンと併用できるシステムとした.
長期にわたり継続的に利用するメンバにとっては,実デバイスによるビーコンは先述の通りバッテリ交換の手間がある.
スマホビーコンはそのようなメンバにとっては,バッテリ交換の手間がないため有用である.
しかしスマートフォンを携帯していないメンバ,スマホビーコンの利用に伴うバッテリ消費が気になるメンバなども想定される.
これらの問題は実デバイスによるビーコンとスマホビーコンのハイブリッド化に
よって解決できる.スマホビーコンで利用する UUID を
実デバイスによるビーコンで利用する UUID と同じ値に設定し同じメンバの在室情報を記録している.
この方法は表\ref{fig:hybrid}に示す通りスマホビーコンか実デバイスによるビーコンの少なくとも片方を携帯していれば記録できるため
継続的にデータを記録する観点から見ても有用である.


\begin{table}[tbh]
  \caption{各ビーコンのみ対応時とハイブリッド対応時の比較}
  \begin{tabular}{|c|c|c|c|c|}
    \hline
                                                             & 長期  メンバ                                                 & 短期・一時的                                                  & スマホ不携帯 & ビーコン不携帯 \\ \hline
    \begin{tabular}[c]{@{}c@{}}実デバイス\\ ビーコンのみ\\\end{tabular} & \begin{tabular}[c]{@{}c@{}}\\△\\ バッテリ交換の手間\end{tabular} & ○                                                       & X      & ○       \\ \hline
    \begin{tabular}[c]{@{}c@{}}スマホ\\ ビーコンのみ\\\end{tabular}   & ○                                                       & \begin{tabular}[c]{@{}c@{}}\\△\\ インストールの手間\end{tabular} & ○      & X       \\ \hline
    \begin{tabular}[c]{@{}c@{}}ハイブリッド\\ 対応\\\end{tabular}    & \begin{tabular}[c]{@{}c@{}}\\○\\\\\end{tabular}         & ○                                                       & ○      & ○       \\ \hline
  \end{tabular}
  \label{fig:hybrid}
\end{table}