\subsection{スマートフォンビーコンと実デバイスによるビーコンのハイブリッド活用}
スマートフォンビーコンのみを利用する場合,様々な状況下でメンバの継続した利用が困難であるため,実デバイスによる
ビーコンと併用できるシステムとした.

ハイブリッド化にあたってスマホビーコンで利用する UUID を
実デバイスによるビーコンで利用する UUID と同じ値に設定しメンバの在室情報を記録している.

長期にわたり継続的に利用するメンバにとっては,実デバイスによるビーコンは先述の通りバッテリ交換の手間がある.
スマートフォンビーコンはそのようなメンバにとっては,バッテリ交換の手間がないため有用である.
しかしスマートフォンを携帯していないメンバ,スマートフォンビーコンの利用に伴うバッテリ消費が気になるメンバなども想定される.

表\ref{fig:hybrid}に示す通り実デバイスによるビーコンとスマートフォンビーコン双方にメリットデメリットが存在する.
それらはメンバの環境によって重要性が異なる.

我々はスマートフォンビーコンと実デバイスによるビーコンのハイブリッド化によってメンバの環境に応じた選択を可能にした.
スマートフォンに我々のアプリを入れたくないメンバやスマートフォンがアプリに対応していないメンバ,
実デバイスによるビーコンを新しく持ち歩きを不安に感じるメンバなど様々な環境に置かれたメンバに対する選択肢の提供によって
利便性が向上しデータの可用性向上に繋がると考えた.

スマートフォンビーコンと実デバイスによるビーコン双方をメンバが受け入れて同時に利用した場合には,
同じUUIDを利用しているため,片方が停止した場合でももう片方で補完が可能である.

この方法は,メンバの利便性を向上できるのみならず,継続的にデータを記録する観点から見ても有用である.


\begin{table}[tbh]
  \caption{各ビーコンのみ対応時とハイブリッド対応時の比較}
  \begin{tabular}{|c|c|c|c|c|}
    \hline
                              & 長期  メンバ              & 短期・一時的              & スマホ不携帯 & ビーコン不携帯 \\ \hline
    \begin{tabular}[c]{@{}c@{}}実デバイス\\ ビーコンのみ\\\end{tabular} & \begin{tabular}[c]{@{}c@{}}\\△\\ バッテリ交換の手間\end{tabular} & ○                         & ×            & ○              \\ \hline
    \begin{tabular}[c]{@{}c@{}}スマホ\\ ビーコンのみ\\\end{tabular} & ○                         & \begin{tabular}[c]{@{}c@{}}\\△\\ インストールの手間\end{tabular} & ○            & ×              \\ \hline
    \begin{tabular}[c]{@{}c@{}}ハイブリッド\\ 対応\\\end{tabular} & \begin{tabular}[c]{@{}c@{}}\\○\\\\\end{tabular} & ○                         & ○            & ○              \\ \hline
  \end{tabular}
  \label{fig:hybrid}
\end{table}