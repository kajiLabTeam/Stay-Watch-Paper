\subsection{スマートフォンビーコンと実デバイスによるビーコンのハイブリッド活用}
スマートフォンビーコンのみを利用する場合,様々な状況下でメンバの継続した利用が困難であるため,実デバイスによる
ビーコンと併用できるシステムとした.
長期にわたり継続的に利用するメンバにとっては,実デバイスによるビーコンは先述の通りバッテリ交換の手間がある.
メンバはその手間からバッテリ交換をしないで放置する場合がある.その場合は在室情報が記録されず可用性が低下する.
スマートフォンビーコンはバッテリ交換を手間に考えて放置するメンバにとっては,バッテリ交換の手間がないため有用である.
しかしスマートフォンを携帯していないメンバ,スマートフォンビーコンの利用に伴うバッテリ消費が気になるメンバなども想定される.

実デバイスによるビーコンとスマートフォンビーコン双方にメリットデメリットが存在する.
それらはメンバの環境によって重要視する点が異なる.
% 例えばスマートフォンビーコンを利用しないメンバとして
% スマートフォンに我々のアプリケーションを入れたくないメンバや
% スマートフォンがアプリに対応していないメンバである.
% 実デバイスによるビーコンを利用しないメンバとして
% 実デバイスによるビーコンの携帯に不安を感じるメンバなどである.
%%%%%%ここから未定ゾーン%%%%%%
スマートフォンにアプリケーションの導入を拒否する理由として,%拒むはどう?
実デバイスによるビーコンで十分だと考えている,
スマートフォンのリソースをBLEビーコンに割り当てたくない,
アプリケーションの導入によるセキュリティリスクを懸念している,
新しい技術に対して不信感を抱いている,
等が挙げられる.
アプリがスマートフォンに対応しておらず利用できない状態として,
対応していないOS及びバージョンを利用している,
スマートフォンの性能が不十分,
ストレージに十分な空き容量がない,
アプリケーションの導入が制限されている端末,
等が挙げられる.
また実デバイスによるビーコンの携帯を拒否する動機として,
スマートフォンビーコンで十分だと考えている,
実デバイスによるビーコンの紛失を懸念している,
実デバイスによるビーコンの携帯を手間に感じている,
ミニマリズムを信奉しており,身の回りに新しいものを増やしたくない,
等が挙げられる.

%%%%%%ここまで未定ゾーン%%%%%%

メンバの思想や所有している端末の都合など様々な要因が存在しており,それらをすべて満たした仕様の策定は困難である.
よってスマートフォンビーコンと実デバイスによるビーコンのハイブリッド化によってメンバに応じた選択を可能にした.
様々な環境の中でも選択肢を提供し,ユーザの利便性向上によってメンバのより長期に渡る継続的な利用が促進できデータの可用性向上につながると考えた.

ハイブリッド化にあたってスマートフォンビーコンで利用するUUIDを実デバイスによるビーコンで利用するUUIDと同じ値に設定しメンバの在室情報を記録している.
スマートフォンビーコンと実デバイスによるビーコン双方をメンバが受け入れて同時に利用した場合,
同じUUIDを利用しているため片方が停止した場合でも,もう片方のビーコンが検知され在室判定ができる.
この方法は,メンバの利便性を向上できるのみならず,継続的にデータを記録する観点から見ても有用である.
実デバイスによるビーコン,スマートフォンビーコンの単独対応とハイブリッド化を表\ref{fig:hybrid}で比較する.

 滞在ウォッチが実デバイスによるビーコンのみに対応していた場合は
長期にわたり継続的に利用するメンバにとってはビーコンの電池交換が手間である.
短期的に利用するメンバにとってはビーコンを携帯するだけで済み手軽である.
しかしビーコンを不注意などで携帯していなかった場合はシステムに記録されない.
スマートフォンビーコンのみに対応していた場合は
長期に渡り継続的に利用するメンバにとってはアプリは手間が少ない.
短期的に利用しするメンバにとってはアプリケーションの登録は手間である.
しかし,スマートフォンが利用できない状態ではシステムに記録されない.
ハイブリッド化をした場合は,それぞれのユースケースに対応できる.

\begin{table}[tbh]
  \caption{各ビーコンのみ対応時とハイブリッド対応時の可用性の比較}
  \begin{tabular}{|c|c|c|c|c|}
    \hline
    \backslashbox{システム}{メンバ}  & 長期  メンバ              & 短期メンバ              & スマホ不携帯 & ビーコン不携帯 \\ \hline
    \begin{tabular}[c]{@{}c@{}}実デバイス\\ ビーコンのみ\\\end{tabular} & \begin{tabular}[c]{@{}c@{}}\\△\\ バッテリ交換の手間\end{tabular} & ○                         & ○            & ×              \\ \hline
    \begin{tabular}[c]{@{}c@{}}スマホ\\ ビーコンのみ\\\end{tabular} & ○                         & \begin{tabular}[c]{@{}c@{}}\\△\\ インストールの手間\end{tabular} & ×            & ○              \\ \hline
    \begin{tabular}[c]{@{}c@{}}ハイブリッド\\ 対応\\\end{tabular} & \begin{tabular}[c]{@{}c@{}}\\○\\\\\end{tabular} & ○                         & ○            & ○              \\ \hline
  \end{tabular}
  \label{fig:hybrid}
\end{table}