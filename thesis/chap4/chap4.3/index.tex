\section*{4.3 スマホアプリによるビーコンと実デバイス}
 我々の利用するシステム「滞在ウォッチ」は現状,実デバイスによるビーコンを利用しておりここではビーコンのアプリケーション化及びそれらのハイブリッド活用について記す.
 実デバイスによるBLEビーコンは低コストでありサイズもコンパクトと携帯が用意である.しかしながら,メンバや管理者にとって利便性が低く結果として可用性が低くなる.
 可用性とは,メンバの在室情報が長期にわたり継続的に記録される能力と定義する.
 
 実デバイスによるビーコン(以下,デバイスビーコン)を利用する場合,バッテリ切れによる問題,初期設定が複雑である問題が存在する.
まずはバッテリ切れによる問題である我々が利用しているビーコンはバッテリとしてCR2016を利用している.
このCR2016は一般的なコイン型リチウム電池であり,またデバイスビーコンにバッテリ切れを通知する機能はない.
バッテリ残量の把握には,ビーコンメーカによる指定のアプリケーションを要し,デバイスビーコン一台ごとに接続する必要がある.
% もっと愚痴を書く




\subsection*{4.3.1 BLEビーコンのアプリケーション}