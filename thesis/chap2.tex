\thispagestyle{myheadings}
\chapter{卒業論文の書式}
\label{sec:format}

\section{論文の書式}
\label{sec:format_thesis}

卒業論文の作成には \LaTeX やWordなどのソフトウェアの使用を原則とする.
通常,論文は2部作成し,1部は情報科学部事務室に提出し,1部は各自が保管
する.事務室に提出する論文は,A4サイズファイルに綴じ,指導教員に確認の
印を押してもらう.綴じ込み用フラットファイルは,1論文につき1冊,学科か
ら各教員へ支給される.ファイルの表紙には論文題目と学籍番号・氏名,指導
教員名を明記する.ファイルを開いたときの第1頁目にも,ファイルの表紙と
同一項目を明記した中表紙をつける.これは,ファイルの外表紙が破損や退色
をしても,内容がわかるようにするためである.

  \begin{tabular}{llcl}
    (1)&提出部数&:&1部(学部事務室に提出)\\
    (2)&用紙    &:&A4サイズ白紙(罫線・枠等なし)\\
    (3)&印字書式& & \\
       &\hspace*{1zw}・印字方向&:&横書き\\
       &\hspace*{1zw}・段組み数&:&1段\\
       &\hspace*{1zw}・文字サイズ&:&10.5ポイント程度\\
       &\hspace*{1zw}・1頁の行数&:&45行程度\\
       &\hspace*{1zw}・頁番号印刷&:&する(ページ外側の上部)\\
       &\hspace*{1zw}・上余白&:&26mm程度\\
       &\hspace*{1zw}・下余白&:&26mm程度\\
       &\hspace*{1zw}・左余白&:&30mm程度\\
       &\hspace*{1zw}・右余白&:&20mm程度\\
    (4)&分量&:&制限なし\\
    (5)&図表&:&・図,表にはそれぞれ通し番号とタイトルをつける\\
          & & &・\underline{表番号およびタイトルは表の上側につける}\\
          & & &・\underline{図番号およびタイトルは図の下側につける}\\
          & & &・グラフには単位および軸の意味を記し,見て理解できるよ
          うにする\\
          & & &・図,表は必ず本文から参照し説明する\\
          & & &\hspace*{1zw}(図,表だけで理解できるのが望ましい)\\
    (6)&見出し&:&・章,節の番号表記,見出しの書き方は別紙に付けたサ
          ンプルに従う\\
          & & &・章が変わるところでは改ページをおこなう\\
          & & &・節が変わるところでは見出しの前に1行あける\\
    (7)&文字種の使い分け&:&・文章中の仮名および漢字は全角とする\\
          & & &・数字,アルファベットは半角とする\\
          & & &・章,節などの見出しは太字などでわかりやすくする\\
          & & &・網掛け,白抜き,倍角等,冗長な強調表現は避ける\\
    (8)&参考文献&:&・参考文献の書き方は付録に示すが,指導教員の
          指示がある場合は,\\
          & & &\hspace*{1zw}それに従うこと\\
    (9)&タイトル等&:&・研究内容を適切に表すタイトルをつける.\\
          & & &\hspace*{1zw}(2行を越えるなど長すぎないように気をつける)\\
          & & &・必要に応じて副題を付けても良いが,この場合も長さに気
                 をつける\\
          & & &・綴じ込み用ファイルにつける外表紙と,中表紙の用意する\\
          & & &\hspace*{1zw}(同じ書式で構わない)\\
          & & &・指導教員から合格を受けたら,中表紙の教員名の右横に印
                 鑑をもらう\\
          & & &・表紙のサンプルを別紙に示す\\
          & & &・目次をつける\\
          & & &・複数人で書いた場合は執筆範囲が担当がわかるようにする\\
  \end{tabular}

\section{論文の構成例}

卒業論文の構成については各指導教員の指示に従うことになるが,参考までに
構成例を次に示す.

\begin{description}
\item[第1章] 「はじめに」または「序論」として,論文概要及び論文の構成
  について説明する.
\item[第2章] 「背景」として論文の背景にあたる内容を書く.
\item[第3章] 「提案手法」として,\textbf{研究の目的}からアプローチまで
  を説明する.章のタイトルが,卒論のタイトルと一致する場合もある.
\item[第4章] 「実験及び考察」提案手法が研究の目的を達成できているかど
  うかを評価確認し,考察する.
\item[第5章] 「まとめと今後の課題」として,達成できたことと,今後の課
  題として取り組むべき内容についてまとめる.
\item[謝辞] お世話になった先生方や先輩達へのお礼を述べる.
\item[参考文献] 卒業論文の執筆に際し,参考にした文献について記述する.
  本文中と\textbf{相互参照}する.
\item[付録] 本文中に入れることが困難であった詳細な定理証明や,アルゴリ
  ズムなどを記述する.
\end{description}
これら,あくまで例であり,各研究室の指導教員の指示に従う.

\section{論文の製本形態と関連資料の整理}
\label{sec:style}

指導教員へ提出する論文の提出形態は指導教員の指示に従う.事務室へ提出す
るものと同等のものを必要とされる場合もあるが,電子的に成果物を提出する
場合もある.また本論文以外のデータやプログラム,作品等についても研究室
に残すものとし,不正等が認められる場合は,厳正な処分が下されることもあ
り得る.
 
\section{レジュメ(概要)の書式}
\label{sec:format_abst}

レジュメの \LaTeX やWordなどのテンプレートは配布されたものを用いて良い.
また,レジュメは電子的にPDFで提出するため,ソフトウェアの種類を限定す
るものではない.

なお,レジュメはテンプレートで示される書式に従うこととするが,伝統的に
は次のような書式が用いられている.ここで示す書式は,一応の目安であり,
1行の文字数や段間の空白については変更が許される.ソフトウェアによって
は,1行の文字数が固定できないものがある.その場合は,ほぼこの書式に準
ずる文字数になるよう,フォントサイズや文字間隔,行間隔を工夫する.

  \begin{tabular}{llcl}
    (1)&提出部数&:&1部(学部事務室に提出)\\
    (2)&用紙    &:&A4サイズ白紙(罫線・枠等なし)\\
    (3)&印字書式& & \\
       &\hspace*{1zw}・印字方向&:&横書き\\
       &\hspace*{1zw}・段組み数&:&2段\\
       &\hspace*{1zw}・表題文字サイズ&:&16ポイント程度\\
       &\hspace*{1zw}・節見出し文字サイズ&:&12ポイント程度\\
       &\hspace*{1zw}・本文文字サイズ&:&10ポイント程度\\
       &\hspace*{1zw}・1頁の行数&:&45行程度\\
       &\hspace*{1zw}・頁番号印刷&:&しない\\
       &\hspace*{1zw}・上余白&:&15mm程度\\
       &\hspace*{1zw}・下余白&:&20mm程度\\
       &\hspace*{1zw}・左余白&:&15mm程度\\
       &\hspace*{1zw}・右余白&:&15mm程度\\
    (4)&分量&:&2頁\\
    (5)&図表&:&・論文の書式に準ずるものとする\\
    (6)&見出し&:&・論文の書式に準ずるが,章が変わるところでは,大見出し
        の前に\\
          & & &\hspace*{1zw}1行あけるものとする\\
    (7)&文字種の使い分け&:&・論文の書式に準ずるものとする\\
    (8)&参考文献&:&・論文の書式に準ずるものとする\\
    (9)&タイトル等&:&・研究内容を適切に表すタイトルをつける.\\
          & & &\hspace*{1zw}(2行を越えるなど長すぎないように気をつける)\\
          & & &・必要に応じて副題を付けても良いが,この場合も長さに気
          をつける\\
          & & &・タイトルに続いて学籍番号,名前をつける\\
          & & &・最後に指導教員名をつける\\
  \end{tabular}

\clearpage

% Local Variables: 
% mode: japanese-LaTeX
% TeX-master: "root"
% End: 
