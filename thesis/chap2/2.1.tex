


\section{屋内位置推定に関する研究}\label{2.1}
% 以前から在室管理は自動化されれば便利なシステムになると言われていた\cite{twitter}.
% 部屋における屋内位置推定にはいくつか在室者の検出手法があり,用途によって人の在否のみと個人を特定する手法がある.
% 滞在ウォッチでは,利用者が目的とする人の居場所を把握できるように,個人を特定する手法に着目する.
% 具体的にはICカードやライブカメラを用いた検出手法があり,これらには導入時に配線工事の手間や高価な機材を必要とするため困難であるのとプライバシへの配慮が必要である.
% そこで,自動で在室者情報を記録する無線通信技術による検出手法に着目する.
% 志毛らのBLEを用いた位置情報共有システムの開発\cite{communication}では,
% 表\ref{tb:positioning}の無線通信による測位技術の比較を行った.
% 消費電力が低く,屋内位置推定に向いている BLEを用いた検出手法が有効だと考えた.


% 屋内位置推定に関する研究はいくつかある.
Wi-Fiを用いた屋内位置推定に関する研究がある\cite{wifi0} \cite{wifi1}.
Wi-Fi測位はWi-Fiアクセスポイントの信号強度を利用して位置推定を行う手法である.
この手法では無線LANのアクセスポイントから発信される電波強度を測定,アクセスポイントとの距離を推定し測位を行う.
Wi-Fi測位のメリットにコストが低い点がある.Wi-Fi技術は多くの所で使用されており,多くの建物にはWi-Fiアクセスポイントが設置されている.そのため追加の設備費用が不要である.
さらにWi-Fi測位の主な測位対象であるスマートフォンやタブレットなどの携帯端末には,基本的にWi-Fiチップが搭載されているため導入が容易である.
欠点はアクセスポイントの設置数によって測位精度が大きく左右される点である.

% \begin{table}[htb]
%   \begin{center}
%     \caption{測位技術の比較}
%     \label{tb:positioning}
%     \begin{tabular}{|l|c|c|c|} \hline
%       測位技術    & 屋外測位 & 屋内測位 & 消費電力 \\ \hline \hline
%       GPS     & ○    & ×    & 高い   \\
%       携帯電話基地局 & △    & ×    & 普通   \\
%       Wi-Fi   & △    & ○    & 普通   \\
%       BLE     & ×    & ◎    & 低い   \\\hline
%     \end{tabular}
%   \end{center}
% \end{table}


カメラを使用した屋内位置推定に関する研究がある\cite{camera0}\cite{camera1}.
カメラを使用し環境中に存在する特徴的なランドマークや形状を識別し,それらを用いて位置を推定する.
具体的には画像認識や3D情報を使った手法がある.
画像認識を使用した手法はカメラが撮影した画像から特徴的な目印となるものを認識し,それらを元に位置を推定する.
特徴的な目印となるものにはQRコード,バーコード,ARマーカーなどが挙げられる.
3D情報を使用した手法は立体的な情報を取得するステレオカメラやRGB-Dカメラを用いて,3D空間内の距離や障害物などから位置を推定する.
この手法では自己の位置を推定するSLAMアルゴリズムや,点群データから環境の詳細な3D情報を抽出する点群処理技術が使用されている.
これらの手法はGPSやWi-Fi測位などの手法と比べて,屋内でも高精度の測位が可能である.
ただしカメラを使用するため人物のプライバシに敏感な情報が含まれる可能性があり,それらに考慮する必要がある.
加えて光の強弱や反射などの影響も受けやすいため,それらが測位精度に影響を与える可能性がある.



磁気指紋法を使用した屋内位置推定に関する研究がある\cite{magnetic0}\cite{magnetic1}.
これは磁気センサを搭載したスマートフォンなどのデバイスを使用して,磁気指紋を測定しそれをもとに屋内の位置を推定する手法である.
磁気指紋とは地球の南極から北極に向かって均一に流れている磁場が観測点の周囲の構造に依存して変化するパターンである.
各地点の磁気の大きさをあらかじめ測定しておき,その情報をもとに磁気指紋データベースを作成する.
磁気センサで取得した磁気の大きさの情報を磁気指紋データベースに照会してその位置を推定する.
ただし地球の磁場は時間ともに変化しており,磁気指紋のデータベースが古くなっている場合,現在の磁場と異なるため正確な位置推定ができない可能性がある.
さらに磁気指紋は建物や鉄骨などの金属構造物の影響を受けるため,それらが存在する場所では正確な位置推定ができない可能性がある.


本研究では狭い範囲でより正確な屋内位置推定が可能である,ビーコンと部屋に設置した受信機を使用した手法を採用する.この手法ならば受信機の設置やビーコンの設定などのコストがかかるがWi-Fi測位や磁気指紋方を使用した手法と比べて,環境に大きく左右されず正確な屋内位置推定が可能である.
またカメラを用いた方法と比較してビーコンを用いた屋内位置推定は,ビーコン単体では個人情報とならないためプライバシーに対するリスクが低い.

















