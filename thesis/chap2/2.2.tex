\section{在室者検出及び在室管理システムに関する研究}\label{2.2}
在室者を検出する手法としてICカードを用いた検出方法
\cite{zyugyou}
\cite{felica}
\cite{densi}
% \cite{felica2}
\cite{notouchic}
\cite{felicaweb}
\cite{suica}
や,ビーコンの受信電波強度を利用した検出方法\cite{bledakoku}\cite{blesuitei}がある.
ICカードを用いた在室者検出方法はメンバにICカードを携帯してもらい,専用の機器などを用いて在室者を検出する手法である.
必要となる機器を導入した後はメンバはICカードを用いるだけで在室記録できる.
% しかし新たに導入する場合は必要となる機器や,ICカードを登録する手間などがありコストは高くなってしまう.
しかしICカードを用いた方法では,メンバが

ビーコンを用いた在室者検出方法には二つある.
1つ目はメンバにビーコンを携帯してもらい,在室者を検出したい部屋に受信機を置く方法がある.
部屋を利用するメンバの在室者検出はビーコンの受信電波を受信機が取得するだけで行えるので,自動で行える.
また,ビーコンはサイズが小さいものが多く,メンバが携帯する負担もかからない.
在室者検出する部屋に受信機を置き,メンバはビーコンを携帯するだけなのでコストも抑えられる.
2つ目にメンバにはスマホを携帯してもらい,在室者を検出したい部屋にビーコンを置く方法がある.
スマホの普及が進んでいる\cite{hukyu}ので,導入コストはビーコンのみであるため,1つ目の方法よりもコストを抑えられる.
しかし,全員がスマホを所持しているわけではないので,スマホを所持していない利用者には別の検出方法を導入する必要がある.

在室者を検出し,在室者情報を管理するシステムに関する研究がある\cite{smartphoneAndRoom}\cite{laboratory}\cite{prep}.
大学や会社では在室者管理システムを用いて,講義の出欠\cite{smartphone}\cite{nfcandroid}\cite{android}\cite{smartbase}\cite{garake}や勤怠管理\cite{amano}が行われている.
スマホを用いて在室者を検出し,在室者情報を管理するシステムは,スマホを所持している人が多いため,新たに必要となる機器の数が少なく,コストを抑えられる\cite{smartphoneAndRoom}.
またビーコンを用いて在室者を検出し,在室者情報を管理するシステムはメンバにビーコンを携帯してもらい,在室者を検出したい部屋に受信機を設置すれば在室者を検出できるので,コストを抑えられる\cite{laboratory}\cite{prep}.
本研究では先ほど述べたようにスマートフォンを所持していないメンバも想定し,メンバ一人一人にビーコンを携帯してもらい,在室者を検出したい部屋に受信機を設置するシステムとした.


