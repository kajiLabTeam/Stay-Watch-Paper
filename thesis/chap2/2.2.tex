\section{在室者の検出方法に関する研究}\label{2.2}
大学や会社では在室者を検出する手法を用いて,講義の出欠\cite{smartphone}\cite{nfcandroid}\cite{android}\cite{smartbase}\cite{garake}や勤怠管理\cite{amano}が行われている.
在室者を検出する手法としてICカードを用いた検出方法
\cite{zyugyou}
\cite{felica}
\cite{densi}
% \cite{felica2}
\cite{notouchic}
\cite{felicaweb}
\cite{suica}
や,ビーコンの受信電波強度を利用した検出方法\cite{bledakoku}\cite{blesuitei}がある.

ICカードを用いた在室者検出方法では利用者にICカードを携帯してもらい,専用の機器などを用いて在室者を検出する手法である.
必要となる機器を導入した後は利用者はICカードを用いるだけで在室者検出ができる.
しかし新たに導入する場合は必要となる機器や,ICカードを登録する手間などコストは高くなってしまう.

ビーコンの受信電波強度を用いた在室者検出方法には二つある.
1つ目は利用者にビーコンを携帯してもらい,在室者を検出したい部屋に受信機を置く方法である.
部屋利用者の在室者検出はビーコンの受信電波を受信機が取得するだけで行えるので,自動で行える.
また,ビーコンはサイズが小さいものが多く,利用者が携帯する負担もかからない.
在室者検出する部屋に受信機を置き,利用者はビーコンを携帯するだけなのでコストも抑えられる.
2つ目に利用者にはスマホを携帯してもらい,在室者を検出したい部屋にビーコンを置く方法である.
スマホの普及が進んでいる\cite{hukyu}ので,導入コストはビーコンのみであるため,1つ目の方法よりもコストを抑えられる.
しかし,全員がスマホを所持しているわけではないので,スマホを所持していない利用者には別の検出方法を導入する必要がある.
本研究では利用者が自発的に在室者情報を記録する手間を必要としない方法として,部屋ごとに受信機を設置し,個人がビーコンを携帯し自動で検出する方法を採用する.

在室者を検出し,在室者情報を管理するシステムに関する研究がある\cite{smartphoneAndRoom}\cite{laboratory}\cite{prep}.
スマホを用いて在室者を検出し,在室者情報を管理するシステムは,スマホを所持している人が多いため,新たに必要となる機器の数が少なく,コストを抑えられる\cite{smartphoneAndRoom}.
またビーコンを用いて在室者を検出し,在室者情報を管理するシステムは利用者にビーコンを携帯してもらい,在室者を検出したい部屋に受信機を設置すれば在室者を検出できるので,コストを抑えられる\cite{laboratory}\cite{prep}.



本研究ではスマートフォンを所持していない利用者も想定し,利用者一人一人にビーコンを携帯してもらい,在室者を検出したい部屋に受信機を設置する方法を採用した.
