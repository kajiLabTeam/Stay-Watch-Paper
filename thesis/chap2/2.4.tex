\section{コミュニケーション促進に関する研究}\label{2.4}
第三者間のコミュニケーションや知っている人同士のコミュニケーションを支援,促進する研究がある.
まず見知らぬ他人や顔だけは知っているがコミュニケーションを取らない人とのコミュニケーションを支援促進するものがある\cite{hati}\cite{tikachat}\cite{compresence}\cite{siruetto}.
気軽に他者とのコミュニケーションができると示唆している.
しかし知っている人同士のコミュニケーションを促進するものではなく,あくまで見知らぬ他者やコミュニケーションを取らない人とのコミュニケーションを支援,促進するものである.
また公共空間でのコミュニケーションを支援,促進する研究がある\cite{komyusoku}\cite{travelingcafe}\cite{meetingpot}\cite{photochat}.
見知らぬ人や顔だけは知っている関係ではなく,同じ空間を共有している人同士のコミュニケーションを支援,促進するものである.
コミュニーションのきっかけや共有できる情報を提示するものとして,本研究でも同じ空間を共有している人同士のコミュニケーションを促進する方法として参考にする.



