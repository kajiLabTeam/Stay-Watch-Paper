%
%  愛知工業大学経営情報科学部情報科学科
%    LaTeXテンプレート(2015.12.22)
%
\documentclass[openany]{jbook}
% 使いたい人は使う
\usepackage{personal}
% 図挿入用
% \usepackage{graphicx}
\usepackage[dvipdfmx]{graphicx}
\usepackage{float}


\usepackage{latexsym}
\usepackage{amssymb}
\usepackage{amsfonts}
\usepackage{ulinej}
\usepackage{url}
\usepackage{mathtools}
\usepackage{here} % 強制的に図を好きな位置に配置するためのパッケージ
\pagestyle{headings}

% カウンタセット
\setcounter{secnumdepth}{3}
\setcounter{tocdepth}{3}
% 定義環境
\newtheorem{definition}{定義}[section]
% 例環境
\newtheorem{example}{例}[section]
% 新しい環境の定義
\newenvironment{indention}[1]{\par
\addtolength{\leftskip}{#1}
\begingroup}{\endgroup\par}
% 関連図書→参考文献
\renewcommand{\bibname}{参考文献}

\topmargin=-14mm
\headsep=15mm
\textwidth=15.7cm
%\baselineskip=22pt
%\renewcommand{\baselinestretch}{1.4}
\textheight=24.5cm  % 33 lines in 1 page
\oddsidemargin=7.5mm
\evensidemargin=7.5mm

% 部分コンパイル用
% \includeonly{title,chap1,chap2,chap3,chap4,chap5,thanks,reference}
\begin{document}

\begin{titlepage}

\ \\
\begin{center}

{\LARGE 愛知工業大学情報科学部情報科学科\\
コンピュータシステム専攻(メディア情報専攻)

\vspace{1.0cm}

令和2年度~卒業論文\\

\vspace{2.0cm}
{\Huge 
\baselineskip=15mm
\textbf{卒業論文の作成・発表マニュアル\\
(情報科学科用)\\}}

\vspace{7.0cm}

2020年2月\\

\vspace{1.0cm}

\begin{tabular}[h]{lll}
  研究者  & K00001 & 愛工太郎\\
         & K00011 & 八草花子\\
         & X00012 & 愛知環状\\
\end{tabular}

\vspace{1.0cm}

指導教員\ \ 情報一郎\ \ 教授}

\end{center}

\end{titlepage}
%%% Local Variables: 
%%% mode: latex
%%% TeX-master: "root"
%%% End: 


%目次を自動的に作る。
\tableofcontents

% 本文






サーバ側には独立したバックエンドシステムとの連携を容易でありWebアプリケーションがバックエンドシステムとの親和性が高く、
より高い可用性とスケーラビリティを実現するができるREST APIを採用した.REST APIは、
複数のクライアントからアクセスができるため,様々なデバイスやプラットフォームからアクセスできる。これにより、より広いユーザー層からアクセスができる
既存の滞在ウォッチのサーバ側のシステムはpythonを用いて構築されており,動的型付け言語なため保守性が低いものであった.
そこで静的型付けであり,高速な処理能力と小さなメモリフットプリントを持つため、Web APIの開発に適しているGolangを採用した.
またGolangは並列処理を容易に実現できるため、高負荷な環境でのWeb APIの開発にも適している。さらに、標準パッケージによるHTTPサーバのサポートを持つため、
Web APIの開発に必要な機能を簡単に実装できる。
これらの特徴から、Golangを使用したWeb APIの開発は、高速でスケーラブルなAPIを提供することができ、開発効率も高い






サーバ側には独立したバックエンドシステムとの連携を容易でありWebアプリケーションがバックエンドシステムとの親和性が高く、
より高い可用性とスケーラビリティを実現するができるREST APIを採用した.REST APIは、
複数のクライアントからアクセスができるため,様々なデバイスやプラットフォームからアクセスできる。これにより、より広いユーザー層からアクセスができる
既存の滞在ウォッチのサーバ側のシステムはpythonを用いて構築されており,動的型付け言語なため保守性が低いものであった.
そこで静的型付けであり,高速な処理能力と小さなメモリフットプリントを持つため、Web APIの開発に適しているGolangを採用した.
またGolangは並列処理を容易に実現できるため、高負荷な環境でのWeb APIの開発にも適している。さらに、標準パッケージによるHTTPサーバのサポートを持つため、
Web APIの開発に必要な機能を簡単に実装できる。
これらの特徴から、Golangを使用したWeb APIの開発は、高速でスケーラブルなAPIを提供することができ、開発効率も高い

% 




サーバ側には独立したバックエンドシステムとの連携を容易でありWebアプリケーションがバックエンドシステムとの親和性が高く、
より高い可用性とスケーラビリティを実現するができるREST APIを採用した.REST APIは、
複数のクライアントからアクセスができるため,様々なデバイスやプラットフォームからアクセスできる。これにより、より広いユーザー層からアクセスができる
既存の滞在ウォッチのサーバ側のシステムはpythonを用いて構築されており,動的型付け言語なため保守性が低いものであった.
そこで静的型付けであり,高速な処理能力と小さなメモリフットプリントを持つため、Web APIの開発に適しているGolangを採用した.
またGolangは並列処理を容易に実現できるため、高負荷な環境でのWeb APIの開発にも適している。さらに、標準パッケージによるHTTPサーバのサポートを持つため、
Web APIの開発に必要な機能を簡単に実装できる。
これらの特徴から、Golangを使用したWeb APIの開発は、高速でスケーラブルなAPIを提供することができ、開発効率も高い






サーバ側には独立したバックエンドシステムとの連携を容易でありWebアプリケーションがバックエンドシステムとの親和性が高く、
より高い可用性とスケーラビリティを実現するができるREST APIを採用した.REST APIは、
複数のクライアントからアクセスができるため,様々なデバイスやプラットフォームからアクセスできる。これにより、より広いユーザー層からアクセスができる
既存の滞在ウォッチのサーバ側のシステムはpythonを用いて構築されており,動的型付け言語なため保守性が低いものであった.
そこで静的型付けであり,高速な処理能力と小さなメモリフットプリントを持つため、Web APIの開発に適しているGolangを採用した.
またGolangは並列処理を容易に実現できるため、高負荷な環境でのWeb APIの開発にも適している。さらに、標準パッケージによるHTTPサーバのサポートを持つため、
Web APIの開発に必要な機能を簡単に実装できる。
これらの特徴から、Golangを使用したWeb APIの開発は、高速でスケーラブルなAPIを提供することができ、開発効率も高い

% 




サーバ側には独立したバックエンドシステムとの連携を容易でありWebアプリケーションがバックエンドシステムとの親和性が高く、
より高い可用性とスケーラビリティを実現するができるREST APIを採用した.REST APIは、
複数のクライアントからアクセスができるため,様々なデバイスやプラットフォームからアクセスできる。これにより、より広いユーザー層からアクセスができる
既存の滞在ウォッチのサーバ側のシステムはpythonを用いて構築されており,動的型付け言語なため保守性が低いものであった.
そこで静的型付けであり,高速な処理能力と小さなメモリフットプリントを持つため、Web APIの開発に適しているGolangを採用した.
またGolangは並列処理を容易に実現できるため、高負荷な環境でのWeb APIの開発にも適している。さらに、標準パッケージによるHTTPサーバのサポートを持つため、
Web APIの開発に必要な機能を簡単に実装できる。
これらの特徴から、Golangを使用したWeb APIの開発は、高速でスケーラブルなAPIを提供することができ、開発効率も高い

\chapter*{謝辞}
\addcontentsline{toc}{chapter}{\protect\numberline {} 謝辞}

本研究を進めるにあたり,多くの御指導,御鞭撻を賜わりました
まず研究の主要な主導者である梶 克彦准教授に深く感謝致します.教授は私の意見を真剣に受け止め,熱心な指導をしてくださりました.また素晴らしい研究室の環境を提供し,私の成長に貢献してくださりました.
次に,日頃から熱心に討論,助言してくださいました
梶研究室のみなさんに深く感謝致します.
さらに家族,友人,同僚にも感謝致します.彼らは私が論文を書くために専念できる環境を整えくれ
私を励ましてくれました.
最後にこの卒業論文を書けたことに感謝します.
研究を通じて多くの学びを得て,自分の人生に大きな影響を与え人間としてより成長できました.


% ここからお遊びゾーン



% Local Variables: 
% mode: latex
% TeX-master: "root"
% End: 



\bibliography{thesis} %hoge.bibから拡張子を外した名前
\bibliographystyle{junsrt} %参考文献出力スタイル

% \thispagestyle{myheadings}
\addcontentsline{toc}{chapter}{\protect\numberline {} 参考文献の書き方例}
    
\begin{thebibliography}{参考文献}

\bibitem{Colby}
Latha, S. Colby and Dirk VanGucht,
``A Grammar Model for Database'',
{\it TECHNICAL REPORT,} NO.282,
June 1989.

\bibitem{Gonnet}
Gaston, H. Gonnet and Frank Wm. Tompa,
``Mind Your Grammar: a New Approach to Modelling Text'',
{\it Proceedings of the 13th VLDB Conference,}Brighton,
pp. 339--346, 1987.

\bibitem{Dzenan}
Dzenan RIDJANOVIC and Micheal L. BRODIE,
``DEFINING DATABASE DYNAMICS WITH ATTRIBUTE GRAMMARS'',
{\it INFORMATION PROCESSING LETTERS,}vol. 14, No. 3,
pp. 132--138, May 1982.


%\bibitem{ifo}
%Abiteboul, S. and Hull, R.
%``IFO: A formal semantic database model''
%{\sl ACM Transactions on Database Systems  12,} 4,
%December 1987,
%pp.525-565.
%
%\bibitem{er}
%Chen, P. P. 
%``The entitiy-relationship model-toward a unified view of data''
%{\sl ACM Transactions on Database Systems 1,} 1,
%March 1976,
%pp.9-36.
%
%\bibitem{sdm}
%Hammer, M. and McLeod, D.
%``Database discription with SDM: A semantic database model''
%{\sl ACM Transactions on Database Systems 6,} 3,
%September 1981,
%pp.351-386.
%
%\bibitem{fdm}
%Shipman, D. W.
%``The functioal data model and the language DAPLEX''
%{\sl ACM Transactions on Database Systems 6,} 1,
%March 1981,
%pp.140-173.
%
%\bibitem{o2}
%Bancilhon, F., et al.
%``The design and implementation of O$_2$, an object-oriented database system''
%{\sl Processings of 2nd International Workshop on Object-Oriented Database Systems}
%1988,
%pp.1-22.
%
%\bibitem{gem}
%Copeland, G., and Maier, D.
%``Making Smalltalk a database system''
%{\sl Processings of the Annual SIGMOD Conference,}
%June 1984.
%
%\bibitem{iris}
%Fishman, D. H., et al. 
%``Iris: An object oriented database management system''
%{\sl ACM Transactions on Database Systems 5,} 1,
%January 1987,
%pp.48-69.
%
%\bibitem{orion}
%Kim, W.,Manerjee, J., Chou, H. T., Garza, J. F., and Woelk, D.
%``Composite object support in an object-oriented database system''
%{\sl Proceedings of OOPSLA,}
%1987,
%pp.118-125.

\bibitem{Hull}
Hull, R. and Yap, C. K.
``The Format model: A theory of database organization'',
{\it JACM 31,}3,
pp. 518--537, 1984.

\bibitem{jap1}
増永良文,
``次世代データベースシステムとしてのオブジェクト指向データベースシステム'',
情報処理,Vol. 32, No. 5,
pp. 490--499, May 1991.

% \bibitem{jap2}
% 田中克己,
% ``オブジェクト指向データベースの基礎概念''
% 情報処理,Vol. 32, No. 5,
% May 1991,
% pp. 500-513.

\bibitem{omt}
J.ランボー M.プラハ W.プレメラニ F.エディ W.ローレンセン,
``OBJECT-ORIENTED MODELING AND DESIGN'',
トッパン, 1992.

% \bibitem{model}
% 米澤明憲,
% ``モデル化と表現''
% 岩波書店(1992).

% \bibitem{prolog}
% 中島秀之,
% ``Prolog''
% 産業図書(1983).

\end{thebibliography}
% Local Variables: 
% mode: japanese-LaTeX
% TeX-master: "root"
% End: 


% これ以降,付録となる
\appendix

% \chapter{論文表紙}
\thispagestyle{myheadings}

\vspace{-1.0cm}

\begin{center}

{\LARGE 愛知工業大学情報科学部情報科学科\\
コンピュータシステム専攻

\vspace{1.0cm}

令和2年度~卒業論文\\

\vspace{2.0cm}

{\Huge 
\baselineskip=15mm
\textbf{独立したコミュニティにおける\\滞在ウォッチの安定運用のための\\システム拡張に関する研究\\}}

\vspace{7.0cm}

2020年2月\\

\vspace{1.0cm}

\begin{tabular}[h]{lll}
  研究者  & K19036 & 亀田優作\\
         & K19074 & 外山瑠起\\
         
\end{tabular}

\vspace{1.0cm}

指導教員\ \ 梶 克彦\ \
准教授}

\end{center}

% Local Variables: 
% mode: latex
% TeX-master: "root"
% End: 



\end{document}

%%% Local Variables: 
%%% mode: latex
%%% TeX-master: t
%%% End: 
