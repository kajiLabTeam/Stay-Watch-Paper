\chapter{おわりに}
\thispagestyle{myheadings}

\section{まとめ}

今回BLEビーコンを用いた在室者検出システム「滞在ウォッチ」を複数間コミュニティで連携するためのシステム拡張を行なった.
滞在ウォッチはBLEビーコンをメンバに持たせて,自動的に行うシステムである.
滞在ウォッチを複数間コミュニティで連携するには,独立したコミュニティで運用できない,長期に渡り在室情報を継続的に記録できない問題を解決した安定運用を行う必要がある.
独立したコミュニティの運用へのアプローチとして利用者の管理とアクセス制御システムの整備を行なった.
その結果,スケーラビリティの高い利用者の管理と適切な範囲での在室情報の共有が可能となった.
長期に渡り在室情報を継続的に記録できない問題へのアプローチとしてスマホアプリによるビーコン化と実デバイスの併用を行なった.
その結果,実デバイスのみで在室管理していた時と比べて,長期に渡り在室情報を記録できるようになった.


\section{今後の課題}

今後の課題としてまずBLEビーコンによる在室判定の誤判定が発生する問題がある.
例えば受信機の設置してある部屋の上の階の部屋にいる人が入室したと判定される場合がある.
この問題への対応策として,信号強度を利用する方法などがある.
信号強度を利用する方法は、受信機が受け取ったビーコンの信号強度を測定し、その部屋にいるかどうかを判定する方法である。信号強度は、送信機から受信機までの距離に比例して減衰する。そのため、受信機が受け取ったビーコンの信号強度が弱い場合は、その部屋にいる可能性が低いと判断できる。
具体的には、受信機が受け取ったビーコンの信号強度を測定し、それが一定のレベル以上であれば、その部屋にいると判断し、そうでなければ、その部屋にいないと判断する。
ただしこの方法は、受信機が受け取ったビーコンの信号強度を基にして、その部屋にいるかどうかを判断するため、部屋の形状や構造によっては、誤った判断をする可能性があり,全ての部屋で正しく判定するには,部屋ごとに応じた閾値を設定する必要があり,コストがかかる.
他の方法としては,BLEの通信規格を使用する方法が考えられる.
ULtra-Wideband(UWB)デバイスやRFIDデバイスを使用した方法がある.
UWBデバイスは,壁を貫通しにくく,高精度な位置測定を行うことができるが,新しい技術であるため,普及が送れており,対応するデバイスやアプリケーションが少ない欠点がある.



今後の課題として単一コミュニティでの運用しかされていないため,実際に複数のコミュニティに導入してもらい,運用を行う必要がある.
運用後はメンバ,管理者,利用者からの意見や得られたデータを元に評価,システムの改善を行う予定である.
また現状のシステムでは複数コミュニティ間でのコミュニケーションを促進する機能の実装には至っていない.
そのため複数コミュニティに向けたイベント開催機能などを実装し運用・評価する必要がある.



% Local Variables: 
% mode: japanese-LaTeX
% TeX-master: "root"
% End: 
