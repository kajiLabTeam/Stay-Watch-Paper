\chapter{おわりに}
\thispagestyle{myheadings}

\section{まとめ}

今回BLEビーコンを用いた在室者検出システム「滞在ウォッチ」を複数間コミュニティで連携するためのシステム拡張を行なった.
滞在ウォッチはBLEビーコンをメンバに持たせて,自動的に行うシステムである.
滞在ウォッチを複数間コミュニティで連携するには,独立したコミュニティで運用できない,長期に渡り在室情報を継続的に記録できない問題を解決した安定運用を行う必要がある.
独立したコミュニティの運用へのアプローチとして利用者の管理とアクセス制御システムの整備を行なった.
その結果,スケーラビリティの高い利用者の管理と適切な範囲での在室情報の共有が可能となった.
長期に渡り在室情報を継続的に記録できない問題へのアプローチとしてスマホアプリによるビーコン化と実デバイスの併用を行なった.
その結果,実デバイスのみで在室管理していた時と比べて,長期に渡り在室情報を記録できるようになった.


\section{今後の課題}


今後の課題として単一コミュニティでの運用しかされていないため,実際に複数のコミュニティに導入してもらい,運用を行う必要がある.
運用後はメンバ,管理者,利用者からの意見や得られたデータを元に評価,システムの改善を行う予定である.
また現状のシステムでは複数コミュニティ間でのコミュニケーションを促進する機能の実装には至っていない.
そのため複数コミュニティに向けたイベント開催機能などを実装し運用・評価する必要がある.



% Local Variables: 
% mode: japanese-LaTeX
% TeX-master: "root"
% End: 
