\chapter{おわりに}\label{5}
\thispagestyle{myheadings}

\section{まとめ}

今回BLEビーコンを用いた在室者検出システム「滞在ウォッチ」を複数間コミュニティで連携するためのシステム拡張を行なった.
滞在ウォッチはBLEビーコンをメンバに持たせて,自動的に動作するシステムである.
滞在ウォッチを複数間コミュニティで連携するには,独立したコミュニティで運用できない,長期に渡り在室情報を継続的に記録できない問題を解決した安定運用を行う必要がある.
独立したコミュニティの運用へのアプローチとして利用者の管理とアクセス制御システムの整備を行なった.
その結果,スケーラビリティの高い利用者の管理と適切な範囲での在室情報の共有が可能となった.
長期に渡り在室情報を継続的に記録できない問題へのアプローチとしてスマホアプリによるビーコン化と実デバイスの併用を行なった.
その結果,実デバイスのみで在室管理していた時と比べて,長期に渡り在室情報を記録できるようになった.


\section{今後の課題}

今後の課題としてまずBLEビーコンによる在室判定の誤判定が発生する問題がある.
例えば受信機の設置してある部屋の上の階の部屋にいる人が入室したと判定される場合である.

この問題への対応策として,信号強度を利用する方法がある.
信号強度を利用する方法は,受信機が受け取ったビーコンの信号強度を測定し,その部屋にいるかどうかを判定する方法である.信号強度は,送信機から受信機までの距離に比例して減衰する.そのため,受信機が受け取ったビーコンの信号強度が弱い場合は,その部屋にいる可能性が低いと判断できる.
具体的には,受信機が受け取ったビーコンの信号強度を測定し,それが一定のレベル以上であれば,その部屋にいると判断し,そうでなければ,その部屋にいないと判断する.
ただしこの方法は,受信機が受け取ったビーコンの信号強度を基にして,その部屋にいるかどうかを判断するため,部屋の形状や構造によっては,誤った判断をする可能性があり,全ての部屋で正しく判定するには,部屋ごとに応じた閾値を設定する必要があり,コストがかかる.

他の方法としては,BLEの以外の通信規格を使用する方法が考えられる.
Ultra-Wideband(UWB)デバイスやZigbeeデバイスを使用した方法がある.
UWBデバイスは,壁を貫通しにくく,高精度な位置測定を行うことができるが,新しい技術であるため,普及が送れており,対応するデバイスやアプリケーションが少ない欠点がある.
Zigbeeデバイスは,ワイヤレス通信技術の一種で,壁を貫通しにくい低効率の電波を使用し範囲が狭いため,隣接する部屋に影響を与えにくいが,伝達速度が
遅いため,大量のデータを伝送する場合には向いていない欠点がある.
UWBデバイスやZigbeeデバイスを使用した方法は上記で述べた欠点もあるが部屋を貫通しにくく,誤判定が減少すると予想されるため,導入を検討する価値があると考える.

次に,BLEビーコンのUUIDの設定に手間がかかる問題がある.現在ビーコンのUUIDをセットアップする際には,専用のアプリを使ってUUIDの書き込みを行う必要がある.
この問題への解決策として,UUIDを書き込むアプリを独自に作成する方法が考えられる.独自のアプリを作成すれば滞在ウォッチサーバ側に対してHTTPリクエストを送りそのレスポンス
としてUUID取得してそれをBLEビーコンに書き込むようなシステムを構築できると考えられる.
BLEビーコンの提供先である株式会社フォーカスシステムズに対してビーコンへの書き込みを行うAPIを提供してもらえないか確認を行ったが,
結果としてそのようなAPIは提供していないとの回答であった.しかし専用のアプリでBLEビーコンへのUUIDの書き込みを行なっているため不可能ではないと考えられる.
そのためBLEビーコンのセットアップの設定の簡略化を行うために,\ref{4.3}章で話したスマホビーコンの機能としてUUIDをBLEビーコンに対して提供する機能の追加の検討を行っていきたい.

プロジェクトの課題として,滞在ウォッチの複数コミュニティ間連携を実際に行い運用を行う必要がある.本研究では,滞在ウォッチが複数コミュニティ間で連携できるように独立したコミュニティにおいて滞在ウォッチの運用ができるように検討を行ったが,実際に複数間コミュニティ間で運用後に発生する問題も考えられる.そのため,運用後はメンバ,管理者,利用者からの意見や得られたデータを元に評価,システムの改善を行う予定である.また現状のシステムでは複数コミュニティ間でのコミュニケーションを促進する機能の実装には至っていない.
そのため複数コミュニティに向けたイベント開催機能などを実装し運用・評価する必要がある.


% 実デバイスによるビーコンに加えてアプリケーションによるビーコンの実装を行ったが,利用にあたって課題が存在する.
% 我々の研究はコミュニティを対象としている.我々のコミュニティは指導教員がおり,研究生がいる.トップダウン構造かつ我々の研究に欠かせないため,他の研究生も協力をしてくれている.
% しかしながら他のコミュニティ,例として企業などではアプリケーションによるビーコンを使う場合,プライベートで利用しているスマートフォンにビーコンアプリを導入することへの倫理的課題が存在する.
% さらに,我々の研究室のように互いの研究のためといったようなモチベーションが存在しないため,スマホアプリによるビーコンの導入を拒否される可能性がある.
% これらの技術外の課題も存在する.

% Local Variables: 
% mode: japanese-LaTeX
% TeX-master: "root"
% End: 
