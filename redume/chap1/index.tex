\thispagestyle{myheadings}

\section{はじめに}
\label{sec:intro}
特定のコミュニティが利用する研究室やコワーキングスペースのような場所(以下,コミュニティスペース)で利用者の在室情報を分析すると様々な応用ができる.
例えば, 部屋の利用者数や時間帯が把握できれば,環境整備や活用状況が少ない部屋の省エネ化の指標となる.
また目的とする人の居場所を把握できれば,コミュニケーションの円滑化や共同作業を支援できる.

しかしコミュニティスペースでは必ずしも在室情報が記録されているとは限らない.
在室者を検出する方法としてスマートフォンやビーコンを用いた検出方法がある [1].
スマートフォンとビーコンを利用し,在室者を検出する手法である.
しかし,部屋利用者が能動的に記録する必要がある.
会社において気軽なコミュニケーション促進を目的とした研究がある [2].
しかし,システムの導入が会社におけるものなのでコミュニティスペースでの環境に適合しないと考えられる.

そこで我々の先行研究として BLE ビーコンを用いた在室管理プラットフォーム「滞在ウォッチ」が提案されている.
滞在ウォッチにはビーコンを持ち歩き在室情報を記録するメンバ,メンバ管理やメンバへのビーコン配布等を行う管理者,
滞在ウォッチの現在状況や履歴を閲覧したりAPIを通して利用する利用者,システムを開発する開発者が存在する.
メンバの負担軽減のために,在室者情報を BLEビーコンで受動的に記録する方法が採用されている.
図1に示すように滞在ウォッチはメンバがビーコンを携帯し部屋ごとに設置された受信機によりビーコンを検出して,在室者管理を自動で行う.
% 在室管理プラットフォームの概要を図 1 に示す.ビーコンを持った利用者が部屋に訪れると受信機が検知し,サーバに在室者情報を送信しデータベースに記録する.
管理者はサーバにメンバの名前とビーコンのIDを登録する.ビーコンは周囲に数秒に 1回電波を発信する.
受信機が検出したビーコンのIDと電波強度はサーバに送信され,入退室した時刻と日時,在室した部屋名がデータベースに記録される.
データベースに保存された情報は独自に作成したAPIによって外部からの利用が可能である.
過去の研究として滞在ウォッチAPIを用いた退勤管理システムや在室状況可視化システム,部屋利用者の来訪促進システム,
コミュニケーション促進システムなど様々な応用システムの構築がされてきた.

本研究の目的は,滞在ウォッチの複数コミュニティ間の連携の実現である.
複数コミュニティ間とは空間的な距離が近いコミュニティスペース間と定義する.
例として,大学の研究室間やビルのオフィス間などが挙げられる。
滞在ウォッチにおける複数コミュニティ間連携によって,普段コミュニケーションを取らない別のコミュニティスペースにいる在室者を把握できる.
コミュニケーションを促進できれば,知見の共有や新規性のある想像ができる可能性がある.

しかし滞在ウォッチは単一コミュニティを前提として設計されているため,複数コミュニティ間の連携にはいくつかの問題点が存在する.
コミュニティで独立した運用ができていない,在室情報を長期にわたり継続的に記録できない点が挙げられる.
上記の問題を解決しなければ複数のコミュニティ間で連携することは難しい.

これらの問題を解決した滞在ウォッチの運用を安定運用と定義する.
本研究では滞在ウォッチを複数コミュニティ間で連携するために独立したコミュニティにおける滞在ウォッチの安定運用のためのシステム拡張について提案する.