
\subsection{ユーザの管理と権限周りのシステムの整備}
従来は単独コミュニティでのみ運用がされており滞在ウォッチのシステムの開発者とユーザの管理を行うユーザが同一であった.
しかし複数コミュニティでこのような運用を行う場合,システム発者の負担が大きく運用するのは難しい.
また在室情報はプライバシーに関わるものであるため複数間コミュニティで適切に扱う必要がある.
そこで Web ページ閲覧者の制限とユーザによる情報公開範囲の変更を行うためにログイン機能の実装を行った.
この場合の制限は管理者が登録したユーザのみ閲覧できるものである.ログイン機能には Firebase Authentication,OAuth2.0,Googleアカウント を使用した認証システムを使用している.ユーザ認証システムを独自で実装するという方法もあるが,パスワードのハッシュ化,ハッシュ化に利用している関数の脆弱性,フォームの改竄リスク等,気をつけなければいけないセキュリティリスクがいくつか存在する.Firebase Authentication を利用した実装ならばそれらのリスクを排除できる.

利用ユーザはGoogleアカウントを用いた認証を行うことで新しくIDとPASSWORDを作ることなくページ閲覧が可能である.まずWeb ページ上のログインボタンを押すと Firebase SDK を利用してリダイレクトを行い図5に示すように Google 認証画面に移動する.ここでユーザが許可すると Google の認可サーバがアクセストークンを発行するための認可コードを発行, Web ページにリダイレクトする.リダイレクト時に付与された認可コードを認可サーバに渡すことでアクセストークンを取得する.
次に Web ページから認証 API に対してリクエストを行う.このリクエストはリクエストを送ったユーザが管理者が登録したユーザであるかを確認するために行う.
Webページ側は取得したアクセストークンをHTTP のリクエストヘッダーに付与してリクエストを送る.
認証 API 側はヘッダーからトークンを取得 Firebase Admin SDK を利用してクライアントから送られてきたトークンが正しいユーザのものであるかの検証を行う.
トークン情報が正しくない場合はサーバ側は ステータスコード401,
トークン情報は正しいがトークンから得られるメールアドレスがデータベースに存在しない場合は ステータスコード 403,トークン情報が正しくトークンから得られるメールアドレスが存在する場合は ステータスコード 200 を返す.
Web ページ側は ステータスコード によって適切なコンテンツの表示を切り替えを行っている.



管理者側はユーザの登録を行う.図6に示すように登録フォームでユーザネームと Gmail アドレス,ユーザロールを入力する.
ユーザネームは Web ページで表示される名前である,Gmail アドレスは閲覧を許可する Google アカウントに紐つくメールアドレス,ユーザロールは登録者の権限レベルを表す.
管理者が登録すると該当ユーザのメールアドレスに対して登録が完了したメールが送信される.
メールを受け取ったユーザはログインするとWebページの閲覧が可能となる.