
\subsection{ユーザの管理と権限周りのシステムの整備}
独立した運用を行うためにはユーザの管理と権限周りのシステムの整備が必要である.
従来は単独コミュニティでのみ運用を前提としており滞在ウォッチのシステムの開発者とユーザの管理を行うユーザが同一であった.
また在室情報はプライバシーに関わるものであるにも関わらず誰でも見れる状態であった.
これは既存の滞在ウォッチはユーザの権限に関する情報とユーザ情報に基づいたWebページログインの仕組みが存在していなかったためである.
しかし複数間コミュニティ間で運用を行う場合,コミュニティの数が増えるに連れてシステム開発者のユーザ管理の負担が大きくなり運用するのは難しくなる.
これを解決するには各コミュニティごとに管理者ユーザを作り,各コミュニティで独立した運用を行う必要がある.
各コミュニティごとに管理者が存在すればコミュニティのユーザの管理を全て行う必要がないためシステム開発者の負担が軽減される.

そこでGoogleアカウントを用いたユーザのログイン機能を実装した.
Googleアカウントを用いたログインのみではGoogleアカウント自体に滞在ウォッチに関する権限情報がないためユーザの識別はできない.
そのためユーザの権限情報とGoogleアカウントを滞在ウォッチデータベースのユーザ情報と紐づけている.
これによりGoogleアカウントでログインしているユーザが管理者ユーザであるかの識別が可能である.



また管理者ユーザがユーザの登録を行えるような仕組みを作成した.
まずユーザはログインした上で管理者ユーザに対して自分のGoogleアカウントを連絡する.
管理者ユーザがWebページのユーザ登録画面からそれを登録することで滞在ウォッチデータベースにユーザ情報とGoogleアカウントが登録される.
ユーザがログインした上でWebページを閲覧する際にWebページ側からユーザのGoogleアカウント情報を滞在ウォッチAPIサーバに送る.
その後滞在ウォッチデータベースにそのGoogleアカウントが登録されているかのを確認する.
登録されている場合はWebページに対してそのユーザの在室情報の閲覧の許可を与える.
つまりGoogleアカウントが滞在ウォッチデータベースのユーザ情報に登録されているかの有無で在室情報の閲覧の可否が決まる.
仮に外部のものがGoogleアカウントを使ってログインしたとしても滞在ウォッチデータベースにそのGoogleアカウントが登録されていないため在室情報の閲覧が不可能である.
これにより適切な範囲での在室情報を扱うことが可能である.