
\subsection{BLEビーコンとスマートフォンのハイブリッド化に対応したシステム}
BLEビーコンのアプリケーション化による可用性向上
ユーザの利便性向上のために,BLEビーコンのアプリケーション化をした.
BLEビーコンの導入は容易だが,運用上の諸問題がある.在室判定精度の向上のために,ビーコンは常に動作する必要がある.
しかしBLEビーコンのバッテリー残量の把握は専用のアプリケーションによる接続を要し,バッテリー切れに気が付かなかったユーザや,電池交換を手間に感じたユーザにバッテリー切れを起こしたビーコンが放置される状況が存在した.
これらの問題は,アプリケーション化に伴いハードウェア面,ソフトウェア面から改善がされた.
ハードウェアとして動作するスマートフォンはユーザが高頻度で状態を確認するため,バッテリー切れなど状況の判別が容易である.
また,スマートフォン自体の可用性を維持するために対策を講じるユーザが多く,ビーコンと比較してハードウェアとして可用性が維持されやすい.
ソフトウェア面では,図7に示す通りスマートフォンの通知領域に動作状況を可視化する.通知領域への表示はビーコンとしての動作と連携しており,アプリケーションの動作中は永続的に表示される.
よってアプリケーションが停止した場合もユーザによる判別が容易であるため,アプリケーション再起動によって可用性が維持されやすい.またバックグラウンド動作によってユーザ操作の負担を低減している.
既存のBLEビーコンの利点として正常に動作している限り,ユーザの操作が不要である点が挙げられる.
アプリケーション化に伴い,ユーザの操作が必要になったがそれを最小限に抑えるため,バックグラウンド動作による負担低減を行った.
ユーザの必要な操作は初回のみ必要なログイン処理とビーコン動作の切り替え処理のみである.ログイン処理ではFirebase Authenticationによって登録されたユーザか検証し,データベースからビーコンのデータを取得する.
ビーコン動作の切り替え処理はユーザの事情に応じた選択肢を提供する.在室情報のデータ可用性の観点からは常時のビーコン動作が望ましいが,同時にユーザの使用するスマートフォンのプライバシー性などへの配慮など倫理的課題が存在する.
それらの観点から図8に示すようにビーコン動作の停止が可能になる動作切り替え処理ボタンを提供する.
また,アプリケーション停止時に自動でビーコン動作の復帰が行えないため,ユーザが通知領域で動作していない状態を確認した場合,自らビーコンの動作を開始できる.