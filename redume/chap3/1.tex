
\subsection{受信機データの精度向上}
在室判定精度向上のために受信機データの精度向上を行った.
以前は1度のスキャン終了後にデータを送信する方法を使用していた.しかしビーコンの発進間隔の都合上,全てのビーコンをスキャンするのが難しい場合が存在した.
そこでまず初めに部屋ごとに設置された受信機が一定時間,周辺機器のスキャンを行う.
その後スキャンを数回繰り返しスキャンされたUUIDごとのRSSIの合計値をUUIDごとにスキャンされた回数で割り平均化して送る形を採用した.
ここでの RSSI の合計値とスキャンされた回数は受信機のデータベースに保存しておりサーバ側にデータを送信する際に全てのデータをクリアする.
毎回送信されたデータをサーバ側で複数回受け取りその値を平均化するという手法もあったが処理の複雑化,サーバ側の過剰な負荷の懸念があったためこの方法を採用した.