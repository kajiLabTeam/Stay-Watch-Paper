%
%  愛知工業大学 情報科学部 情報科学科
%    要旨集用LaTeXテンプレート(2016.11.28)
%
%  original by Nobuhiro Ito
%  revised by Susumu Suzuki & Masashi Morimoto on 2016.11.28
%
\documentclass{jarticle}
\pagestyle{empty}

%% レイアウト
\setlength{\topmargin}{-10.4mm}
\setlength{\headheight}{0mm}
\setlength{\headsep}{0mm}
\setlength{\textheight}{262mm}
\setlength{\textwidth}{180mm}
%\setlength{\topskip}{7mm}
\setlength{\evensidemargin}{-10.4mm} 
\setlength{\oddsidemargin}{-10.4mm} 
\setlength{\columnsep}{8mm}

%% パッケージ環境
% graphicx: 画像読み込み
%  dvipdfmxをドライバ指定することでpdf/png/jpeg形式の図を利用可能
%  異なるドライバを利用する場合はそのドライバ指定に変更
% \usepackage{graphicx} % suzuki
\usepackage[dvipdfmx]{graphicx} % suzuki
% subcaption: subfigureを置き換えたパッケージ(複数の図用)
% \setlength{\footskip}{12mm}
% \usepackage{subfigure}	 % suzuki
\usepackage{subcaption} % suzuki
% color:文章に色をつけたいとき
%  利用例:\textcolor{red}{文章}
% \usepackage{color}
\usepackage{booktabs}
\usepackage{setspace}

% 行間調整
\setstretch{0.9}

%sectionのフォントサイズ修正
\makeatletter
\def\section{\@startsection {section}{1}{\z@}{2.5ex plus -1ex minus -.2ex}{1.3 ex plus .1ex}{\large\bf}}
\makeatother 

%subsectionのフォントサイズ修正
\makeatletter
\def\subsection{\@startsection {subsection}{1}{\z@}{1.5ex plus -1ex minus -.4ex}{0.3 ex plus .1ex}{\bf}}
\makeatother 

\begin{document}
\twocolumn[

  \begin{center}
    %タイトル
    {\LARGE \textbf{独立したコミュニティにおける滞在ウォッチの安定運用のためのシステム拡張に関する研究}}\\
    %サブタイトル
    %{\Large \textbf{必要に応じてサブタイトル}}
  \end{center}

  \begin{center}
    % 著者
    \begin{tabular}{cccc}
      % 1名の場合
      %\multicolumn{4}{c}{K11001 愛工総和}\\
      % 2名の場合
      %& K11002 愛工七音 & X11003 愛工頼音 &\\
      % 3名の場合
      %K11001 愛工総和 & K11002 愛工今鹿 & X11003 愛工姫星&\\
      % 4名の場合
      K19074 外山瑠起 & k190 亀田優作 \\
      % 指導教員
      \multicolumn{4}{c}{\textbf{指導教員} 梶克彦}
    \end{tabular}
    \hspace{2zw}
  \end{center}
]

%--------------------------------------------
\section{スマホアプリによるビーコンと実デバイス}\label{4.3}
我々の利用するシステム「滞在ウォッチ」は実デバイスによるビーコンを利用している.
ここではビーコンのアプリケーション化及びそれらのハイブリッド活用について記す.

初めに滞在ウォッチで使用しているBLEビーコンの仕様について解説する.
BLEビーコンとはBLE規格のペリフェラル(Peripheral)動作を用いて,周囲にアドバタイズを行うデバイスとセントラル(Central)動作を用いて周囲をスキャンするデバイスによって構成される.
ペリフェラル動作とは通信を受け付ける子機としての動作である.
セントラル動作によってスキャンしたデバイスに反応しGATT(Generic attribute profile)内に保持したサービスやキャラクタリスティックのデータの送受信を行う.
キャラクタリスティックとは,そのデバイスが保持するサービスやデータであり,今回の研究では個人識別符号として利用するUUIDが該当する.
UUID(Universally Unique Identifier)とは,オブジェクトを一位に識別するために重複がない前提で用いる128bitの数値である.
キャラクタリスティックのデータとしてこれをビーコンは保持している.






\subsection*{4.3.1 スマホビーコン}

実デバイスによるビーコンはメンバの利便性が低く可用性に問題がある.
可用性とは,メンバの在室情報が長期にわたり継続的に記録される能力と定義する.
実デバイスによるビーコンを利用する場合,高い可用性を維持するにはバッテリ交換に配慮する必要がある.
そこで既存研究ではバッテリ切れが発生した場合,管理者がメンバにSlackを用いて
通知しバッテリ交換を催促していた.
しかし交換されない状況が存在した.
これは通知による催促が不確実かつ即時性がないためである.
通知は一定期間の在室がない場合にバッテリ切れの可能性があると見做して通知している.
そのため通知の正確性が低い上,バッテリ切れに対してタイムラグがある.
また交換作業がメンバに委ねられており,その手間による利便性が低くバッテリ切れの放置が発生した.



上記の問題のアプローチとしてメンバの利便性を向上させるため,スマホアプリによるビーコン動作(以下,スマホビーコン)を行った.
BLEビーコンの代替としてスマートフォンを利用可能にするとバッテリ交換の手間が削減される.
またスマートフォンユーザにとってスマートフォンはコミュニケーションツールとしての用途からバッテリ切れを配慮する傾向が強い.
よって実デバイスによるビーコンと比べてスマホビーコンはバッテリが維持されやすく利便性が向上すると考えた.

スマホビーコンは基本的にバックグラウンドに常駐させる利用法を想定し実装した.
既存研究では,メンバに実デバイスによるビーコンを携帯させ,能動的な記録動作の必要がない.
バックグラウンドに常駐させる方式は実デバイスによるビーコンと同様に能動的な記録動作を必要としないため同等の利便性がある.
スマートフォンの画面表示が可能な利点を利用し,スマートフォンの通知領域に動作状況を表示した.
通知領域への表示はスマホビーコンの動作と連携しており,動作中に表示される.
メンバにとって実デバイスによるビーコンは動作の把握が困難であったが,通知領域への表示により動作の把握が可能になった.
そのためビーコン動作の停止に気が付きやすく,メンバによる再起動が行われた場合,可用性の向上が期待できる.
\subsection{スマートフォンビーコンと実デバイスによるビーコンのハイブリッド活用}
スマートフォンビーコンのみを利用する場合,様々な状況下でメンバの継続した利用が困難であるため,実デバイスによる
ビーコンと併用できるシステムとした.
長期にわたり継続的に利用するメンバにとっては,実デバイスによるビーコンは先述の通りバッテリ交換の手間がある.
メンバはその手間からバッテリ交換をしないで放置する場合がある.その場合は在室情報が記録されず可用性が低下する.
スマートフォンビーコンはバッテリ交換を手間に考えて放置するメンバにとっては,バッテリ交換の手間がないため有用である.
しかしスマートフォンを携帯していないメンバ,スマートフォンビーコンの利用に伴うバッテリ消費が気になるメンバなども想定される.

実デバイスによるビーコンとスマートフォンビーコン双方にメリットデメリットが存在する.
それらはメンバの環境によって重要性が異なる.
例えばスマートフォンビーコンを利用しないメンバとして
スマートフォンに我々のアプリケーションを入れたくないメンバやスマートフォンがアプリに対応していないメンバ,
実デバイスによるビーコンを新しく持ち歩きを不安に感じるメンバなどである.
メンバの思想や所有している端末の都合など様々な要因が存在しており,それらをすべて満たした仕様の策定は困難である.
よってスマートフォンビーコンと実デバイスによるビーコンのハイブリッド化によってメンバの環境に応じた選択を可能にした.
様々な環境の中でも選択肢を提供し,ユーザの利便性を向上することでメンバのより積極的な参加を促すことができデータの可用性向上につながると考えた.

ハイブリッド化にあたってスマホビーコンで利用するUUIDを実デバイスによるビーコンで利用するUUIDと同じ値に設定しメンバの在室情報を記録している.
スマートフォンビーコンと実デバイスによるビーコン双方をメンバが受け入れて同時に利用した場合,
同じUUIDを利用しているため片方が停止した場合でも,もう片方で補完が可能である.
この方法は,メンバの利便性を向上できるのみならず,継続的にデータを記録する観点から見ても有用である.
実デバイスによるビーコン,スマートフォンビーコンの単独対応とハイブリッド化を表\ref{fig:hybrid}で比較する.
 滞在ウォッチが実デバイスによるビーコンのみに対応していた場合は
長期にわたり継続的に利用するメンバにとってはビーコンの電池交換が手間である.
短期的に利用するメンバにとってはビーコンを携帯するだけで済み手軽である.
しかしビーコンを不注意などで携帯していなかった場合はシステムに記録されない.
スマートフォンビーコンのみに対応していた場合は
長期に渡り継続的に利用するメンバにとってはアプリは手間が少ない.
短期的に利用しするメンバにとってはアプリケーションの登録は手間である.
しかし,スマートフォンが利用できない状態ではシステムに記録されない.
ハイブリッド化をした場合は,それぞれのユースケースに対応できる.

\begin{table}[tbh]
  \caption{各ビーコンのみ対応時とハイブリッド対応時の比較}
  \begin{tabular}{|c|c|c|c|c|}
    \hline
                              & 長期  メンバ              & 短期メンバ              & スマホ不携帯 & ビーコン不携帯 \\ \hline
    \begin{tabular}[c]{@{}c@{}}実デバイス\\ ビーコンのみ\\\end{tabular} & \begin{tabular}[c]{@{}c@{}}\\△\\ バッテリ交換の手間\end{tabular} & ○                         & ○            & ×              \\ \hline
    \begin{tabular}[c]{@{}c@{}}スマホ\\ ビーコンのみ\\\end{tabular} & ○                         & \begin{tabular}[c]{@{}c@{}}\\△\\ インストールの手間\end{tabular} & ×            & ○              \\ \hline
    \begin{tabular}[c]{@{}c@{}}ハイブリッド\\ 対応\\\end{tabular} & \begin{tabular}[c]{@{}c@{}}\\○\\\\\end{tabular} & ○                         & ○            & ○              \\ \hline
  \end{tabular}
  \label{fig:hybrid}
\end{table}







\begin{figure}[tbh]
  \centering
  \includegraphics[width=8cm]{image/StayWatch.jpg}
  \caption{「滞在ウォッチ」の概要図}
  \label{multipleBPM}
\end{figure}

%--------------------------------------------
\section{スマホアプリによるビーコンと実デバイス}\label{4.3}
我々の利用するシステム「滞在ウォッチ」は実デバイスによるビーコンを利用している.
ここではビーコンのアプリケーション化及びそれらのハイブリッド活用について記す.

初めに滞在ウォッチで使用しているBLEビーコンの仕様について解説する.
BLEビーコンとはBLE規格のペリフェラル(Peripheral)動作を用いて,周囲にアドバタイズを行うデバイスとセントラル(Central)動作を用いて周囲をスキャンするデバイスによって構成される.
ペリフェラル動作とは通信を受け付ける子機としての動作である.
セントラル動作によってスキャンしたデバイスに反応しGATT(Generic attribute profile)内に保持したサービスやキャラクタリスティックのデータの送受信を行う.
キャラクタリスティックとは,そのデバイスが保持するサービスやデータであり,今回の研究では個人識別符号として利用するUUIDが該当する.
UUID(Universally Unique Identifier)とは,オブジェクトを一位に識別するために重複がない前提で用いる128bitの数値である.
キャラクタリスティックのデータとしてこれをビーコンは保持している.






\subsection*{4.3.1 スマホビーコン}

実デバイスによるビーコンはメンバの利便性が低く可用性に問題がある.
可用性とは,メンバの在室情報が長期にわたり継続的に記録される能力と定義する.
実デバイスによるビーコンを利用する場合,高い可用性を維持するにはバッテリ交換に配慮する必要がある.
そこで既存研究ではバッテリ切れが発生した場合,管理者がメンバにSlackを用いて
通知しバッテリ交換を催促していた.
しかし交換されない状況が存在した.
これは通知による催促が不確実かつ即時性がないためである.
通知は一定期間の在室がない場合にバッテリ切れの可能性があると見做して通知している.
そのため通知の正確性が低い上,バッテリ切れに対してタイムラグがある.
また交換作業がメンバに委ねられており,その手間による利便性が低くバッテリ切れの放置が発生した.



上記の問題のアプローチとしてメンバの利便性を向上させるため,スマホアプリによるビーコン動作(以下,スマホビーコン)を行った.
BLEビーコンの代替としてスマートフォンを利用可能にするとバッテリ交換の手間が削減される.
またスマートフォンユーザにとってスマートフォンはコミュニケーションツールとしての用途からバッテリ切れを配慮する傾向が強い.
よって実デバイスによるビーコンと比べてスマホビーコンはバッテリが維持されやすく利便性が向上すると考えた.

スマホビーコンは基本的にバックグラウンドに常駐させる利用法を想定し実装した.
既存研究では,メンバに実デバイスによるビーコンを携帯させ,能動的な記録動作の必要がない.
バックグラウンドに常駐させる方式は実デバイスによるビーコンと同様に能動的な記録動作を必要としないため同等の利便性がある.
スマートフォンの画面表示が可能な利点を利用し,スマートフォンの通知領域に動作状況を表示した.
通知領域への表示はスマホビーコンの動作と連携しており,動作中に表示される.
メンバにとって実デバイスによるビーコンは動作の把握が困難であったが,通知領域への表示により動作の把握が可能になった.
そのためビーコン動作の停止に気が付きやすく,メンバによる再起動が行われた場合,可用性の向上が期待できる.
\subsection{スマートフォンビーコンと実デバイスによるビーコンのハイブリッド活用}
スマートフォンビーコンのみを利用する場合,様々な状況下でメンバの継続した利用が困難であるため,実デバイスによる
ビーコンと併用できるシステムとした.
長期にわたり継続的に利用するメンバにとっては,実デバイスによるビーコンは先述の通りバッテリ交換の手間がある.
メンバはその手間からバッテリ交換をしないで放置する場合がある.その場合は在室情報が記録されず可用性が低下する.
スマートフォンビーコンはバッテリ交換を手間に考えて放置するメンバにとっては,バッテリ交換の手間がないため有用である.
しかしスマートフォンを携帯していないメンバ,スマートフォンビーコンの利用に伴うバッテリ消費が気になるメンバなども想定される.

実デバイスによるビーコンとスマートフォンビーコン双方にメリットデメリットが存在する.
それらはメンバの環境によって重要性が異なる.
例えばスマートフォンビーコンを利用しないメンバとして
スマートフォンに我々のアプリケーションを入れたくないメンバやスマートフォンがアプリに対応していないメンバ,
実デバイスによるビーコンを新しく持ち歩きを不安に感じるメンバなどである.
メンバの思想や所有している端末の都合など様々な要因が存在しており,それらをすべて満たした仕様の策定は困難である.
よってスマートフォンビーコンと実デバイスによるビーコンのハイブリッド化によってメンバの環境に応じた選択を可能にした.
様々な環境の中でも選択肢を提供し,ユーザの利便性を向上することでメンバのより積極的な参加を促すことができデータの可用性向上につながると考えた.

ハイブリッド化にあたってスマホビーコンで利用するUUIDを実デバイスによるビーコンで利用するUUIDと同じ値に設定しメンバの在室情報を記録している.
スマートフォンビーコンと実デバイスによるビーコン双方をメンバが受け入れて同時に利用した場合,
同じUUIDを利用しているため片方が停止した場合でも,もう片方で補完が可能である.
この方法は,メンバの利便性を向上できるのみならず,継続的にデータを記録する観点から見ても有用である.
実デバイスによるビーコン,スマートフォンビーコンの単独対応とハイブリッド化を表\ref{fig:hybrid}で比較する.
 滞在ウォッチが実デバイスによるビーコンのみに対応していた場合は
長期にわたり継続的に利用するメンバにとってはビーコンの電池交換が手間である.
短期的に利用するメンバにとってはビーコンを携帯するだけで済み手軽である.
しかしビーコンを不注意などで携帯していなかった場合はシステムに記録されない.
スマートフォンビーコンのみに対応していた場合は
長期に渡り継続的に利用するメンバにとってはアプリは手間が少ない.
短期的に利用しするメンバにとってはアプリケーションの登録は手間である.
しかし,スマートフォンが利用できない状態ではシステムに記録されない.
ハイブリッド化をした場合は,それぞれのユースケースに対応できる.

\begin{table}[tbh]
  \caption{各ビーコンのみ対応時とハイブリッド対応時の比較}
  \begin{tabular}{|c|c|c|c|c|}
    \hline
                              & 長期  メンバ              & 短期メンバ              & スマホ不携帯 & ビーコン不携帯 \\ \hline
    \begin{tabular}[c]{@{}c@{}}実デバイス\\ ビーコンのみ\\\end{tabular} & \begin{tabular}[c]{@{}c@{}}\\△\\ バッテリ交換の手間\end{tabular} & ○                         & ○            & ×              \\ \hline
    \begin{tabular}[c]{@{}c@{}}スマホ\\ ビーコンのみ\\\end{tabular} & ○                         & \begin{tabular}[c]{@{}c@{}}\\△\\ インストールの手間\end{tabular} & ×            & ○              \\ \hline
    \begin{tabular}[c]{@{}c@{}}ハイブリッド\\ 対応\\\end{tabular} & \begin{tabular}[c]{@{}c@{}}\\○\\\\\end{tabular} & ○                         & ○            & ○              \\ \hline
  \end{tabular}
  \label{fig:hybrid}
\end{table}







\section{スマホアプリによるビーコンと実デバイス}\label{4.3}
我々の利用するシステム「滞在ウォッチ」は実デバイスによるビーコンを利用している.
ここではビーコンのアプリケーション化及びそれらのハイブリッド活用について記す.

初めに滞在ウォッチで使用しているBLEビーコンの仕様について解説する.
BLEビーコンとはBLE規格のペリフェラル(Peripheral)動作を用いて,周囲にアドバタイズを行うデバイスとセントラル(Central)動作を用いて周囲をスキャンするデバイスによって構成される.
ペリフェラル動作とは通信を受け付ける子機としての動作である.
セントラル動作によってスキャンしたデバイスに反応しGATT(Generic attribute profile)内に保持したサービスやキャラクタリスティックのデータの送受信を行う.
キャラクタリスティックとは,そのデバイスが保持するサービスやデータであり,今回の研究では個人識別符号として利用するUUIDが該当する.
UUID(Universally Unique Identifier)とは,オブジェクトを一位に識別するために重複がない前提で用いる128bitの数値である.
キャラクタリスティックのデータとしてこれをビーコンは保持している.






\subsection*{4.3.1 スマホビーコン}

実デバイスによるビーコンはメンバの利便性が低く可用性に問題がある.
可用性とは,メンバの在室情報が長期にわたり継続的に記録される能力と定義する.
実デバイスによるビーコンを利用する場合,高い可用性を維持するにはバッテリ交換に配慮する必要がある.
そこで既存研究ではバッテリ切れが発生した場合,管理者がメンバにSlackを用いて
通知しバッテリ交換を催促していた.
しかし交換されない状況が存在した.
これは通知による催促が不確実かつ即時性がないためである.
通知は一定期間の在室がない場合にバッテリ切れの可能性があると見做して通知している.
そのため通知の正確性が低い上,バッテリ切れに対してタイムラグがある.
また交換作業がメンバに委ねられており,その手間による利便性が低くバッテリ切れの放置が発生した.



上記の問題のアプローチとしてメンバの利便性を向上させるため,スマホアプリによるビーコン動作(以下,スマホビーコン)を行った.
BLEビーコンの代替としてスマートフォンを利用可能にするとバッテリ交換の手間が削減される.
またスマートフォンユーザにとってスマートフォンはコミュニケーションツールとしての用途からバッテリ切れを配慮する傾向が強い.
よって実デバイスによるビーコンと比べてスマホビーコンはバッテリが維持されやすく利便性が向上すると考えた.

スマホビーコンは基本的にバックグラウンドに常駐させる利用法を想定し実装した.
既存研究では,メンバに実デバイスによるビーコンを携帯させ,能動的な記録動作の必要がない.
バックグラウンドに常駐させる方式は実デバイスによるビーコンと同様に能動的な記録動作を必要としないため同等の利便性がある.
スマートフォンの画面表示が可能な利点を利用し,スマートフォンの通知領域に動作状況を表示した.
通知領域への表示はスマホビーコンの動作と連携しており,動作中に表示される.
メンバにとって実デバイスによるビーコンは動作の把握が困難であったが,通知領域への表示により動作の把握が可能になった.
そのためビーコン動作の停止に気が付きやすく,メンバによる再起動が行われた場合,可用性の向上が期待できる.
\subsection{スマートフォンビーコンと実デバイスによるビーコンのハイブリッド活用}
スマートフォンビーコンのみを利用する場合,様々な状況下でメンバの継続した利用が困難であるため,実デバイスによる
ビーコンと併用できるシステムとした.
長期にわたり継続的に利用するメンバにとっては,実デバイスによるビーコンは先述の通りバッテリ交換の手間がある.
メンバはその手間からバッテリ交換をしないで放置する場合がある.その場合は在室情報が記録されず可用性が低下する.
スマートフォンビーコンはバッテリ交換を手間に考えて放置するメンバにとっては,バッテリ交換の手間がないため有用である.
しかしスマートフォンを携帯していないメンバ,スマートフォンビーコンの利用に伴うバッテリ消費が気になるメンバなども想定される.

実デバイスによるビーコンとスマートフォンビーコン双方にメリットデメリットが存在する.
それらはメンバの環境によって重要性が異なる.
例えばスマートフォンビーコンを利用しないメンバとして
スマートフォンに我々のアプリケーションを入れたくないメンバやスマートフォンがアプリに対応していないメンバ,
実デバイスによるビーコンを新しく持ち歩きを不安に感じるメンバなどである.
メンバの思想や所有している端末の都合など様々な要因が存在しており,それらをすべて満たした仕様の策定は困難である.
よってスマートフォンビーコンと実デバイスによるビーコンのハイブリッド化によってメンバの環境に応じた選択を可能にした.
様々な環境の中でも選択肢を提供し,ユーザの利便性を向上することでメンバのより積極的な参加を促すことができデータの可用性向上につながると考えた.

ハイブリッド化にあたってスマホビーコンで利用するUUIDを実デバイスによるビーコンで利用するUUIDと同じ値に設定しメンバの在室情報を記録している.
スマートフォンビーコンと実デバイスによるビーコン双方をメンバが受け入れて同時に利用した場合,
同じUUIDを利用しているため片方が停止した場合でも,もう片方で補完が可能である.
この方法は,メンバの利便性を向上できるのみならず,継続的にデータを記録する観点から見ても有用である.
実デバイスによるビーコン,スマートフォンビーコンの単独対応とハイブリッド化を表\ref{fig:hybrid}で比較する.
 滞在ウォッチが実デバイスによるビーコンのみに対応していた場合は
長期にわたり継続的に利用するメンバにとってはビーコンの電池交換が手間である.
短期的に利用するメンバにとってはビーコンを携帯するだけで済み手軽である.
しかしビーコンを不注意などで携帯していなかった場合はシステムに記録されない.
スマートフォンビーコンのみに対応していた場合は
長期に渡り継続的に利用するメンバにとってはアプリは手間が少ない.
短期的に利用しするメンバにとってはアプリケーションの登録は手間である.
しかし,スマートフォンが利用できない状態ではシステムに記録されない.
ハイブリッド化をした場合は,それぞれのユースケースに対応できる.

\begin{table}[tbh]
  \caption{各ビーコンのみ対応時とハイブリッド対応時の比較}
  \begin{tabular}{|c|c|c|c|c|}
    \hline
                              & 長期  メンバ              & 短期メンバ              & スマホ不携帯 & ビーコン不携帯 \\ \hline
    \begin{tabular}[c]{@{}c@{}}実デバイス\\ ビーコンのみ\\\end{tabular} & \begin{tabular}[c]{@{}c@{}}\\△\\ バッテリ交換の手間\end{tabular} & ○                         & ○            & ×              \\ \hline
    \begin{tabular}[c]{@{}c@{}}スマホ\\ ビーコンのみ\\\end{tabular} & ○                         & \begin{tabular}[c]{@{}c@{}}\\△\\ インストールの手間\end{tabular} & ×            & ○              \\ \hline
    \begin{tabular}[c]{@{}c@{}}ハイブリッド\\ 対応\\\end{tabular} & \begin{tabular}[c]{@{}c@{}}\\○\\\\\end{tabular} & ○                         & ○            & ○              \\ \hline
  \end{tabular}
  \label{fig:hybrid}
\end{table}






%--------------------------------------------

\section{今後の課題}
今後の課題として現段階では1つのコミュニティでの運用しかされていないため,実際に複数のコミュニティに導入してもらい,運用を行う必要がある.運用後は,ユーザからの意見や得られたデータを元にシステムの改善を行う予定である.また現状のシステムではコミュニケーションを促進するような仕組みがないためその仕組みづくりを行いたい.

%--------------------------------------------
\begin{thebibliography}{9}
  \bibitem{ura}
  浦正広, 遠藤守, 山田雅之, 宮崎慎也, 岩崎公弥子, 毛利勝廣, 安田孝美, ``天文教育に向けたスマートフォンの活用による対話型の星座検索モデルの提案'', 情報文化学会誌, Vol. 19, No. 2, pp.42--49, 2012.

  \bibitem{tex}
  奥村晴彦, 黒木祐介, ``改訂第6版 \LaTeX2$\epsilon$ 美文書作成入門'', 技術評論社, 2013.

\end{thebibliography}
\end{document}

%%% Local Variables: 
%%% mode: japanese-latex
%%% TeX-master: t
%%% End: 
